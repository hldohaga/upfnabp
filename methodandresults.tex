% Options for packages loaded elsewhere
\PassOptionsToPackage{unicode}{hyperref}
\PassOptionsToPackage{hyphens}{url}
%
\documentclass[
]{article}
\usepackage{amsmath,amssymb}
\usepackage{lmodern}
\usepackage{iftex}
\ifPDFTeX
  \usepackage[T1]{fontenc}
  \usepackage[utf8]{inputenc}
  \usepackage{textcomp} % provide euro and other symbols
\else % if luatex or xetex
  \usepackage{unicode-math}
  \defaultfontfeatures{Scale=MatchLowercase}
  \defaultfontfeatures[\rmfamily]{Ligatures=TeX,Scale=1}
\fi
% Use upquote if available, for straight quotes in verbatim environments
\IfFileExists{upquote.sty}{\usepackage{upquote}}{}
\IfFileExists{microtype.sty}{% use microtype if available
  \usepackage[]{microtype}
  \UseMicrotypeSet[protrusion]{basicmath} % disable protrusion for tt fonts
}{}
\makeatletter
\@ifundefined{KOMAClassName}{% if non-KOMA class
  \IfFileExists{parskip.sty}{%
    \usepackage{parskip}
  }{% else
    \setlength{\parindent}{0pt}
    \setlength{\parskip}{6pt plus 2pt minus 1pt}}
}{% if KOMA class
  \KOMAoptions{parskip=half}}
\makeatother
\usepackage{xcolor}
\usepackage[margin=1in]{geometry}
\usepackage{graphicx}
\makeatletter
\def\maxwidth{\ifdim\Gin@nat@width>\linewidth\linewidth\else\Gin@nat@width\fi}
\def\maxheight{\ifdim\Gin@nat@height>\textheight\textheight\else\Gin@nat@height\fi}
\makeatother
% Scale images if necessary, so that they will not overflow the page
% margins by default, and it is still possible to overwrite the defaults
% using explicit options in \includegraphics[width, height, ...]{}
\setkeys{Gin}{width=\maxwidth,height=\maxheight,keepaspectratio}
% Set default figure placement to htbp
\makeatletter
\def\fps@figure{htbp}
\makeatother
\setlength{\emergencystretch}{3em} % prevent overfull lines
\providecommand{\tightlist}{%
  \setlength{\itemsep}{0pt}\setlength{\parskip}{0pt}}
\setcounter{secnumdepth}{5}
\newlength{\cslhangindent}
\setlength{\cslhangindent}{1.5em}
\newlength{\csllabelwidth}
\setlength{\csllabelwidth}{3em}
\newlength{\cslentryspacingunit} % times entry-spacing
\setlength{\cslentryspacingunit}{\parskip}
\newenvironment{CSLReferences}[2] % #1 hanging-ident, #2 entry spacing
 {% don't indent paragraphs
  \setlength{\parindent}{0pt}
  % turn on hanging indent if param 1 is 1
  \ifodd #1
  \let\oldpar\par
  \def\par{\hangindent=\cslhangindent\oldpar}
  \fi
  % set entry spacing
  \setlength{\parskip}{#2\cslentryspacingunit}
 }%
 {}
\usepackage{calc}
\newcommand{\CSLBlock}[1]{#1\hfill\break}
\newcommand{\CSLLeftMargin}[1]{\parbox[t]{\csllabelwidth}{#1}}
\newcommand{\CSLRightInline}[1]{\parbox[t]{\linewidth - \csllabelwidth}{#1}\break}
\newcommand{\CSLIndent}[1]{\hspace{\cslhangindent}#1}
\usepackage{multirow}
\usepackage{multicol}
\usepackage{colortbl}
\usepackage{hhline}
\newlength\Oldarrayrulewidth
\newlength\Oldtabcolsep
\usepackage{longtable}
\usepackage{float}
\usepackage{wrapfig}
\usepackage{array}
\usepackage{hyperref}
\ifLuaTeX
  \usepackage{selnolig}  % disable illegal ligatures
\fi
\IfFileExists{bookmark.sty}{\usepackage{bookmark}}{\usepackage{hyperref}}
\IfFileExists{xurl.sty}{\usepackage{xurl}}{} % add URL line breaks if available
\urlstyle{same} % disable monospaced font for URLs
\hypersetup{
  pdftitle={bp upf and na},
  hidelinks,
  pdfcreator={LaTeX via pandoc}}

\title{bp upf and na}
\author{}
\date{\vspace{-2.5em}}

\begin{document}
\maketitle

{
\setcounter{tocdepth}{2}
\tableofcontents
}
\hypertarget{bp-and-upf-and-na-in-ndns-dissertation}{%
\section*{BP and UPF and Na in NDNS
Dissertation}\label{bp-and-upf-and-na-in-ndns-dissertation}}
\addcontentsline{toc}{section}{BP and UPF and Na in NDNS Dissertation}

\hypertarget{trends-in-the-association-between-ultra-processed-foods-salt-intake-and-blood-pressure-using-the-national-dietary-and-nutritional-survey-ndns-in-england-2008-2019}{%
\section*{Trends in the association between Ultra-processed foods, salt
intake and Blood Pressure using the National Dietary and Nutritional
Survey (NDNS) in England
2008-2019}\label{trends-in-the-association-between-ultra-processed-foods-salt-intake-and-blood-pressure-using-the-national-dietary-and-nutritional-survey-ndns-in-england-2008-2019}}
\addcontentsline{toc}{section}{Trends in the association between
Ultra-processed foods, salt intake and Blood Pressure using the National
Dietary and Nutritional Survey (NDNS) in England 2008-2019}

\hypertarget{david-ohagan}{%
\section*{David O'Hagan}\label{david-ohagan}}
\addcontentsline{toc}{section}{David O'Hagan}

200299857

\hypertarget{dissertation-submitted-in-partial-fulfilment-of-the-requirements-for-the-degree-of-master-of-public-health-the-university-of-liverpool}{%
\subsection*{Dissertation submitted in partial fulfilment of the
requirements for the degree of Master of Public Health, The University
of
Liverpool}\label{dissertation-submitted-in-partial-fulfilment-of-the-requirements-for-the-degree-of-master-of-public-health-the-university-of-liverpool}}
\addcontentsline{toc}{subsection}{Dissertation submitted in partial
fulfilment of the requirements for the degree of Master of Public
Health, The University of Liverpool}

\hypertarget{august-2023}{%
\subsection*{August 2023}\label{august-2023}}
\addcontentsline{toc}{subsection}{August 2023}

\newpage

\hypertarget{dedication}{%
\subsection*{Dedication}\label{dedication}}
\addcontentsline{toc}{subsection}{Dedication}

To Julie Andrew and Sophie

\newpage

\hypertarget{acknowledgments}{%
\subsection*{Acknowledgments}\label{acknowledgments}}
\addcontentsline{toc}{subsection}{Acknowledgments}

To Zoe and Martn

To Paul

\newpage

\newpage

\hypertarget{abstract}{%
\section*{Abstract}\label{abstract}}
\addcontentsline{toc}{section}{Abstract}

This is a secondary data study looking at BP and the effects of UPF and
Na.

Data from the national dietary and nutrition survey (NDNS) was
interrogated to establish any correlation between UPF intake and BP. The
role of sodium was also examined.

The study showed that there is a correlation between UPF intake and BP,
which disappears when Age is a covariable. It also showed that sodium
intake has no correlation with BP as an isolated variable, but that it
is important within multivariable models.

The importance of this is that BP is a clinical outcome, and a proxy
measure for CVD. The study shows a complex relationship between UPF
intake and population risk of BP and CVD. It also shows that reduction
of the sodium content may be effective at reducing the overall risk.

Policy should aim to reduce percentage intake of UPF and increase
percentage intake of unprocessed foods.

\newpage

\hypertarget{introduction}{%
\section{Introduction}\label{introduction}}

There is thought to be an association between blood pressure (BP) and
intake of Ultra-processed Foods (UPF) and Salt intake. This study looks
at this association in the data set of the National dietary and
nutrition survey (NDNS).

The study gives data from 2008 to 2019. There have been changes in the
intake of UPF, of salt and BP over that time.

This study will describe those changes.

As part of this description, I will identify how BP is affected by these
key dietary factors. I will attempt to identify the role salt intake
plays in the effect of ultraprocessed food on BP.

\hypertarget{public-health-impact}{%
\subsubsection{Public Health Impact}\label{public-health-impact}}

Public Health intends to reduce the burden of ill health across the
population. BP is an indicator of the health of the population, in that
it is a risk factor for a number of non-communicable diseases (NCD).

Dietary approaches to improving public health are able to deliver
proportionate and universal interventions to populations to reduce the
incidence of NCD. These can be delivered up stream at the policy level.

This is effective and efficient and minimises cost.

They also can be used at an individual level, which does risk them being
used for victim blaming.

The commercial and social determinants of health play out a significant
role in research, and delivery of public health improvements around
food.

\hypertarget{epistemology}{%
\subsubsection{Epistemology}\label{epistemology}}

The epistemological approach of this study is positivist. I use a
quantitative approach in a mechanistic and deterministic model. However,
I am aware that this model is an incomplete description of the whole of
reality. So that whilst I work within this positivist framework, I am
aware that the model is limited by the isolation which defines the
parameters of the study.

In particular I am aware that real world application to dietary change
requires interaction with social and economic factors. These factors are
often much better understood within critical realist and social
constructionist models. The commercial and social determinants of health
are both constructionist models which have a great deal of impact on the
reality of dietary effects on BP and on the availability of UPF and on
their nutritional constituents.

\hypertarget{positionality}{%
\subsubsection{Positionality}\label{positionality}}

In a positivist paradigm the observer is external to the model.
Acknowledging that there are constructivist aspects to this study allows
that the observer is closer to the model. My positionality is therefore
of interest to interpretation of the derived model, but also to
understand reasons for decisions about the approach to the data. I share
with Jafar {[}@jafarWhatPositionalityShould2018{]} an intention to lead
in describing my positionality in this quantitative study.

I am from a biomedical background, which brings an attachment to
positivist ideals. However, as a practising physician I am aware of the
interaction of any number of social factors on the health of
participants as Evans and Trotter
{[}@evansEpistemologyUncertaintyPrimary2009{]} discuss . These impact on
food `choices', which might be determined by social expectations as much
as by income, or geography. They also impact on `hard' clinical
measurements such as BP. This can be affected by position and room
temperature as well as by the relationship between the observer and the
participant.

This work is primarily to complete requirements for an MPH degree which
means that it is influenced by factors around health equity and classic
epidemiology as taught on the course. It is produced in collaboration
with a research group with a long established reputation in food
research in public health, which may steer the results in a conservative
direction.

Positivist `grand isolation' may reduce the influence of these factors,
but they remain as influences.

I accept that to proceed, whilst I need to be aware of the limitations
of the positivist approach and the necessity of making pragmatic
selections that there is some degree of validity to the resulting
dataset. Otherwise, analysis of it would be of no purpose.

\newpage

\hypertarget{literature-review}{%
\section{Literature Review}\label{literature-review}}

\hypertarget{introduction-1}{%
\subsection{Introduction}\label{introduction-1}}

This section will describe the search strategy and techniques used to
identify articles to make up the review. Then there will be a review of
separate sections of the literature, before developing a synthesis of
the literature at present.

The search strategy has a core systematic approach but is augmented with
additional items from a range of sources. The success of the search is
that it identifies a wide variety of articles which help to outline and
augment the argument developed. There will be a critical review of the
literature, and synthesis, with expanation on how it relates to the
research question.

\hypertarget{search-strategy}{%
\subsection{Search Strategy}\label{search-strategy}}

My search of the literature identifies the majority of articles of
relevance. Starting with a broad search strategy, the results are
narrowed identifying those of particular relevance, by reading abstracts
and cross referencing with other papers. After discussion colleagues
passed on further relevant literature.

Also, I identify papers from the bibliographies of identified papers.
Reviews and meta-analyses are good at presenting search strategies and
identifying high value studies.

These identify search terms not initially included. Many of these
searches consider broader clinical endpoints, using metabolic syndrome,
diabetes and cerebrovascular and cardiovascular disease.

My search terms are included in the table below. They were searched
through a meta database which includes Medline, and Ovid and Scopus.
This produced 1328 results the search allowed medical, public health,
nursing articles to be prioritised and engineering, chemical, and
technology articles to be deprioritised.

There were no time limits, language limits or availability limits in the
initial search. These 1328 were reduced down by reading titles and
abstracts to identify relevant articles.

Table 1: Table of search terms used

(``ultra-processed food'' OR ``ultra-processed foods'' OR
``ultraprocessed food'' OR ``ultraprocessed foods'' OR ``ultra-processed
product'' OR ``ultra-processed products'' OR ``ultra-processing'' OR
``food processing'' OR ``processed food'' OR ``processed foods'' OR
``NOVA'' OR ``NOVA system'' OR ``NOVA food classification'' OR ``NOVA
classification system'') AND (hypertension OR ``high blood pressure'' OR
``high blood pressures'' OR ``blood pressure'' OR ``systolic pressure''
OR ``diastolic pressure'' OR ``systolic blood pressure'' OR ``diastolic
blood pressure'') AND (adult OR adults OR aged OR ``middle aged'' OR
elderly OR ``older adult'')

\hypertarget{search-results}{%
\subsection{Search results}\label{search-results}}

The results demonstrate the recent introduction of the NOVA
classification (1,2). The Ultraprocessed category comes from this. Prior
to this most papers try to identify specific nutritional deficits.
Dickie at al (3) show modifications and review the working of the NOVA
system.

Clinical outcomes in studies are varied. There are many studies which
use multiple outcomes. Other studies use self-declared clinical states.

Papers were excluded which related to technology including food
technology. They were also excluded if the primary purpose of the paper
was unrelated to dietary or nutritional causes of clinical outcomes.

\begin{itemize}
\tightlist
\item
  1 describe literature
\item
  2 synthesise literature
\item
  3 critique literature
\item
  4 explain role of study within context
\end{itemize}

\hypertarget{bp-ncd-and-public-health}{%
\subsection{BP, NCD and Public Health~}\label{bp-ncd-and-public-health}}

Non-communicable disease is an increasing burden on public health.~
Blood pressure is a significant contributor to NCD. Blood pressure rises
are described as idiopathic as their cause is not clear.~ Dietary and
lifestyle causes are sought as explanations, if true this still is
difficult to narrow down. Social and commercial determinants of health
have a significant role to play.~ ~ In reviewing the Framingham study,
Kannel (1,2) describes how risk factor medicine came about. He describes
raised blood pressure as a `prominent member' of a group of risks in
cardiovascular disease. A disease which is the outcome of `multiple
forces'. His description sees this as part of the march of progress in
understanding cardiovascular disease in particular, but also
non-communicable disease. He identifies that cardiologists alone cannot
conquer cardiovascular disease. Since then NCD and in particular BP has
come to feature more and more. With studies that showed that reducing BP
reduced the risk of CVD. This placed Blood pressure detection,
management, and control at the centre of reducing CVD (3--7).

The causes of BP, as Kannel explains, are divided into secondary BP
where there is an identified pathological cause and `essential' or
idiopathic BP where no cause is identifiable. Contributors to and
partial causes of this essential BP have been sought, at individual and
societal levels, using medical and epidemiological approaches (8--10).
Key factors are often separated into lifestyle causes (11), and social
determinants(12--17). Commerce also has a role to play in a causation
model which embraces an understanding of causation on a population
scale.

Lifestyle factors are contented. Whilst individual choice is involved.
The range of choices available to individuals is limited by the nature
of their society. A misapplication of lifestyle results in blaming
individuals for the poor choices determined by their social and
commercial environment.

Instead of trying to change activity of millions of people can be more
effective to change laws and policies once (17--22). These `upstream'
changes are relatively simple, and are much more effective though they
can also be reversed (23). Opposition sometimes comes from industry.

\hypertarget{ph-and-bp}{%
\subsection{PH and BP}\label{ph-and-bp}}

\begin{enumerate}
\def\labelenumi{\arabic{enumi}.}
\tightlist
\item
  Campos-Nonato I, Vargas Meza J, Nieto C, Ariza AC, Barquera S.
  Reducing sodium consumption in mexico: A strategy to decrease the
  morbidity and mortality of cardiovascular diseases. Frontiers in
  public health. 2022;10:857818--857818.
\item
  Cappuccio FP, Capewell S. Facts, Issues, and Controversies in Salt
  Reduction for the Prevention of Cardiovascular Disease. 2015;7(1):21.
\item
  Carmen BS, Karl V, Michelle G, Johnson N, LeeAnna L, Beryl W, et
  al.~The UnProcessed pantry project (UP3). Family \& community health.
  2022;45(1):23--33.
\item
  Colombet Z, Schwaller E, Head A, Kypridemos C, Capewell S, O'Flaherty
  M. OP12 Social inequalities in ultra-processed food intakes in the
  United Kingdom: A time trend analysis (2008--2018). J Epidemiol
  Community Health. 2022 Aug 1;76(Suppl 1):A6--7.
\item
  Colombet Z, Simioni M, Drogue S, Lamani V, Perignon M, Martin-Prevel
  Y, et al.~Demographic and socio-economic shifts partly explain the
  Martinican nutrition transition: an analysis of 10-year health and
  dietary changes (2003--2013) using decomposition models. Public health
  nutrition. 2021;24(18):6323--34.
\item
  Institute of Medicine, Board on Population Health and Public Health
  Practice, Food and Nutrition Board, Committee on the Consequences of
  Sodium Reduction in Populations, Oria M, Yaktine AL, et al.~Sodium
  Intake in Populations: Assessment of Evidence {[}Internet{]}.
  Washington, D.C., UNITED STATES: National Academies Press; 2013
  {[}cited 2023 Jan 21{]}. Available from:
  \url{http://ebookcentral.proquest.com/lib/liverpool/detail.action?docID=3379068}
\item
  Institute of Medicine, Food and Nutrition Board, Committee on
  Strategies to Reduce Sodium Intake, Boon CS, Taylor CL, Henney JE.
  Strategies to Reduce Sodium Intake in the United States
  {[}Internet{]}. Washington, D.C., UNITED STATES: National Academies
  Press; 2010 {[}cited 2023 Jan 21{]}. Available from:
  \url{http://ebookcentral.proquest.com/lib/liverpool/detail.action?docID=3378676}
\item
  Iso H, Shimamoto T, Yokota K, Ohki M, Sankai T, Kudo M, et
  al.~{[}Changes in 24-hour urinary excretion of sodium and potassium in
  a community-based heath education program on salt reduction{]}. Nihon
  Koshu Eisei Zasshi. 1999 Oct;46(10):894--903.
\item
  Ji C, Cappuccio FP. Socioeconomic inequality in salt intake in Britain
  10 years after a national salt reduction programme. BMJ Open. 2014 Aug
  26;4(8):e005683--e005683.
\item
  Ji C, Cappuccio FP. Socioeconomic inequality in salt intake in Britain
  10 years after a national salt reduction programme. BMJ Open. 2014 Aug
  1;4(8):e005683.
\item
  Jones NR, Tong TY, Monsivais P. Meeting UK dietary recommendations is
  associated with higher estimated consumer food costs: an analysis
  using the National Diet and Nutrition Survey and consumer expenditure
  data, 2008--2012. Public Health Nutrition. 2018 Apr;21(5):948--56.
\item
  Kannel WB. Hypertension: Reflections on Risks and Prognostication. Med
  Clin North Am. 2009 May;93(3):541-Contents.
\item
  Laverty AA, Link to external site this link will open in a new window,
  Kypridemos C, Seferidi P, Vamos EP, Pearson-Stuttard J, et
  al.~Quantifying the impact of the Public Health Responsibility Deal on
  salt intake, cardiovascular disease and gastric cancer burdens:
  interrupted time series and microsimulation study. Journal of
  Epidemiology and Community Health. 2019 Sep;73(9):881.
\item
  Leeuw E de, Simos J, editors. Healthy cities: the theory, policy, and
  practice of value-based urban planning. New York, NY: Springer; 2017.
  515 p.
\item
  MacGregor GA, He FJ, Pombo-Rodrigues S. Food and the responsibility
  deal: how the salt reduction strategy was derailed. BMJ. 2015 Apr
  28;350:h1936.
\item
  Mahmood SS, Levy D, Vasan RS, Wang TJ. The Framingham Heart Study and
  the epidemiology of cardiovascular disease: a historical perspective.
  The Lancet. 2014 Mar 15;383(9921):999--1008.
\item
  Millett C, Laverty AA, Stylianou N, Bibbins-Domingo K, Pape UJ.
  Impacts of a National Strategy to Reduce Population Salt Intake in
  England: Serial Cross Sectional Study. PLoS One. 2012 Jan
  4;7(1):e29836.
\item
  Moreira PVL, Baraldi LG, Moubarac JC, Monteiro CA, Newton A, Capewell
  S, et al.~Comparing Different Policy Scenarios to Reduce the
  Consumption of Ultra-Processed Foods in UK: Impact on Cardiovascular
  Disease Mortality Using a Modelling Approach. Hernandez AV, editor.
  PLoS ONE. 2015 Feb 13;10(2):e0118353.
\item
  National Food Strategy, editor. National Food Strategy\,:\,: part one.
  {[}Internet{]}. London\,: National Food Strategy,; 2020. Available
  from: \url{https://www.nationalfoodstrategy.org/partone/}
\item
  WHO. High blood pressure: a public health problem {[}Internet{]}.
  World Health Organization - Regional Office for the Eastern
  Mediterranean. {[}cited 2022 Jan 19{]}. Available from:
  \url{http://www.emro.who.int/media/world-health-day/public-health-problem-factsheet-2013.html}
\end{enumerate}

\hypertarget{salt-and-bp}{%
\subsection{Salt and BP~}\label{salt-and-bp}}

Salt is a contributor to the physiology of BP~ Its role in pathology is
less clear.~ There are increasing levels of intake.~~ This is correlated
with increasing BP readings.~ Other nutrients have also been
correlated.~ ~ The role of salt in normal and abnormal BP control is
established (23--25). However there remain areas of contention(26).
There may be individuals with higher sensitivity to salt (27).
Understanding the best approaches to reducing salt is difficult. Is it
best to get individuals to reduce intake(28--32), or for all of the food
industry to reduce salt levels(23,33).

\hypertarget{upf-and-bp}{%
\subsection{UPF and BP~}\label{upf-and-bp}}

Nova classification looks at food beyond the nutrient level.~~ It
incorporates ideas relating to `processing of food'~~ But also includes
availability and intake which are all affected. Increasing Category four
or UPF is associated with increasing BP.~ ~Other approaches to food
classification try to address more than the nutritional content. There
is always conflict between commercial interests and restriction to the
freedom to exploitation

Food classification has traditionally concentrated on nutritional
analysis eg Nutriscore (34--37). The social aspect of food has been
studied famously by Bourdieu (38,39). The effect of the social and
commercial nature of food is partly accounted for in Monteiro's Nova
classification. Dickie et al(35,40) tried to develop a system which took
this idea further, but struggled to build a model which was any more
effective.

Monteiro's initial explanation uses the concept of `processing'
(41--47). This idea separates foods into categories based on the amount
of processing that occurs before the food is consumed. Group one are
foods which are in a natural state, as plucked from the tree. Group two
is foods which are used in processes to modify group one foods. Group
three initially was all other foods, but was soon separated into
minimally processed foods, and group four the ultra-processed foods.

Explanations for the differential effect of these foods have developed
as quickly as new ultra-processed foods have been developed . Is it due
to nutritional content(48)? They are high in salt and sugar on average.
Is it due to effects on satiety, or changes to appetite(49)? Is it due
to being easy to buy, and easy to eat(50)? Is it because they don't
require time and effort in the home to process? Is it because these
processes are industrial? Is it because these foods contain `chemicals'
or new ingredients? These explanations move from nutritional through
into social and commercial.

All these critiques are possible because of the social element to the
classification. Nutrition based classifications appear less socially
divisive due to scientific isolation. They still contain elements of
social factors. In particular, the way that foods are analysed can
change their reported nutritional content. Eg a `standard' food may be
compared to a `traditionally prepared' food. The first is prepared in a
factory with control of its nutrition, the second by a home cook with
limited access to nutrition modification technology.

Statements about the scheme often discuss the high salt and sugar
content. Papers discussing the effect on physiology, and pathology in
particular highlight these, but they do not back their statements with
analysis. They do not show that the sodium, and UPF together increase
the risk of CVD, or BP rise. This dissertation intends to address this
gap

\begin{enumerate}
\def\labelenumi{\arabic{enumi}.}
\tightlist
\item
  A. A, Gan HJ, M.Y. H, K. KS, Zainudin AA. Food classification system
  based on food processing and its relationship with nutritional status
  of adults in Terengganu, Malaysia. Food Research. 2019;4(2):539--46.
\item
  Aceves-Martins M, Bates RL, Craig LCA, Chalmers N, Horgan G, Boskamp
  B, et al.~Nutritional Quality, Environmental Impact and Cost of
  Ultra-Processed Foods: A UK Food-Based Analysis. IJERPH. 2022 Mar
  8;19(6):3191.
\item
  Aceves-Martins M, Link to external site this link will open in a new
  window, Bates RL, Link to external site this link will open in a new
  window, Craig LCA, Chalmers N, et al.~Nutritional Quality,
  Environmental Impact and Cost of Ultra-Processed Foods: A UK
  Food-Based Analysis. International journal of environmental research
  and public health {[}Internet{]}. 2022 {[}cited 2022 Oct 28{]};19(6).
  Available from:
  \url{http://www.proquest.com/publiccontent/docview/2644005015?pq-origsite=primo}
\item
  Armendariz M, Pérez-Ferrer C, Basto-Abreu A, Lovasi GS, Bilal U,
  Barrientos-Gutiérrez T. Changes in the Retail Food Environment in
  Mexican Cities and Their Association with Blood Pressure Outcomes. Int
  J Environ Res Public Health. 2022 Jan 26;19(3):1353.
\item
  Astrup A, Monteiro CA. Does the concept of ``ultra-processed foods''
  help inform dietary guidelines, beyond conventional classification
  systems? Debate consensus. The American Journal of Clinical Nutrition.
  2022 Dec 1;116(6):1489--91.
\item
  Astrup A, Monteiro CA. Does the concept of ``ultra-processed foods''
  help inform dietary guidelines, beyond conventional classification
  systems? Debate consensus \textbar{} The American Journal of Clinical
  Nutrition \textbar{} Oxford Academic. The American Journal of Clinical
  Nutrition {[}Internet{]}. 2022 Oct 17 {[}cited 2022 Oct 25{]};
  Available from:
  \url{https://academic-oup-com.liverpool.idm.oclc.org/ajcn/advance-article/doi/10.1093/ajcn/nqac230/6762413}
\item
  Bawajeeh A, Zulyniak M, Evans C, Cade J. P21 Taste classification of
  foods consumed in the national diet and nutrition survey. J Epidemiol
  Community Health. 2021 Sep 1;75(Suppl 1):A52.
\item
  Colombet Z, Schwaller E, Head A, Kypridemos C, Capewell S, O'Flaherty
  M. OP12 Social inequalities in ultra-processed food intakes in the
  United Kingdom: A time trend analysis (2008--2018). J Epidemiol
  Community Health. 2022 Aug 1;76(Suppl 1):A6--7.
\item
  Cuj M, Grabinsky L, Yates-Doerr E. Cultures of Nutrition:
  Classification, Food Policy, and Health. Medical Anthropology. 2021
  Jan 2;40(1):79--97.
\item
  D'avila HF, Kirsten VR. Energy intake from ultra-processed foods among
  adolescents. Revista paulista de pediatria. 2017;35(1):54--60.
\item
  Dickie S, Woods J, Machado P, Lawrence M. Nutrition Classification
  Schemes for Informing Nutrition Policy in Australia: Nutrient-Based,
  Food-Based, or Dietary-Based? Curr Dev Nutr. 2022 Jul 4;6(8):nzac112.
\item
  Dickie S, Woods J, Machado P, Lawrence M. A novel food
  processing-based nutrition classification scheme for guiding policy
  actions applied to the Australian food supply. Frontiers in Nutrition
  {[}Internet{]}. 2023 {[}cited 2023 Feb 10{]};10. Available from:
  \url{https://www.frontiersin.org/articles/10.3389/fnut.2023.1071356}
\item
  Gupta D, Khanal P, Khan M. Sustainability and ultra-processed foods:
  role of youth. Sustainability, agri, food and environmental research.
  2021;
\item
  Hodge A. In this issue: Ultra-processed food and health. Public health
  nutrition. 2021;24(11):3177--8.
\item
  Mertens E, Colizzi C, Peñalvo JL. Ultra-processed food consumption in
  adults across Europe. Eur J Nutr. 2022;61(3):1521--39.
\item
  Monteiro CA, Moubarac JC, Cannon G, Ng SW, Popkin B. Ultra-processed
  products are becoming dominant in the global food system. Obesity
  Reviews. 2013;14(S2):21--8.
\item
  Monteiro CA, Astrup A. Does the concept of ``ultra-processed foods''
  help inform dietary guidelines, beyond conventional classification
  systems? YES. The American Journal of Clinical Nutrition. 2022 Dec
  1;116(6):1476--81.
\item
  Monteiro CA. Nutrition and health. The issue is not food, nor
  nutrients, so much as processing. Public Health Nutrition. 2009
  May;12(5):729--31.
\item
  Monteiro CA, Cannon G, Levy R, Moubarac JC, Jaime P, Martins AP, et
  al.~NOVA. The star shines bright. World Nutrition. 2016 Jan
  7;7(1--3):28--38.
\item
  Monteiro CA, Levy RB, Claro RM, Castro IRR de, Cannon G. A new
  classification of foods based on the extent and purpose of their
  processing. Cad Saúde Pública. 2010 Nov;26:2039--49.
\item
  Monteiro CA, Levy RB, Claro RM, Castro IRR de, Cannon G. Uma nova
  classificação de alimentos baseada na extensão e propósito do seu
  processamento. Cad Saúde Pública. 2010 Nov;26:2039--49.
\item
  Muñoz-Lara A, Moncada-Patiño J, Tovar-Vega A, Aguilar-Zavala H. THE
  CONSUMPTION OF ULTRA-PROCESSED FOODS, ANTHROPOMORPHIC MEASUREMENTS AND
  BLOOD CHEMISTRY IN MEXICAN SCHOOL-AGE CHILDREN. Annals of nutrition
  and metabolism. 2020;76:212-.
\item
  Rauber F, Louzada ML da C, Steele EM, Rezende LFM de, Millett C,
  Monteiro CA, et al.~Ultra-processed foods and excessive free sugar
  intake in the UK: a nationally representative cross-sectional study.
  BMJ Open. 2019 Oct 1;9(10):e027546.
\item
  Rauber F, Steele EM, Louzada ML da C, Millett C, Monteiro CA, Levy RB.
  Ultra-processed food consumption and indicators of obesity in the
  United Kingdom population (2008-2016). Meyre D, editor. PLoS ONE. 2020
  May 1;15(5):e0232676.
\item
  Southall JR. Ultra-processed food consumption linked to risk for
  colorectal cancer among men. HEM/ONC Today. 2022 Oct 25;23(14):13.
\item
  Vargas-Meza J, Cervantes-Armenta MA, Campos-Nonato I, Nieto C,
  Marrón-Ponce JA, Barquera S, et al.~Dietary sodium and potassium
  intake: Data from the mexican national health and nutrition survey
  2016. Nutrients. 2022;14(2):281-.
\item
  Wang L, Du M, Wang K, Khandpur N, Rossato SL, Drouin-Chartier JP, et
  al.~Association of ultra-processed food consumption with colorectal
  cancer risk among men and women: results from three prospective US
  cohort studies. BMJ. 2022 Aug 31;378:e068921.
\item
  Wang L, Martínez Steele E, Du M, Pomeranz JL, O'Connor LE, Herrick KA,
  et al.~Trends in Consumption of Ultraprocessed Foods Among US Youths
  Aged 2-19 Years, 1999-2018. JAMA. 2021 Aug 10;326(6):519--30.
\item
  Weinstein G, Vered S, Ivancovsky‐Wajcman D, Zelber‐Sagi S,
  Ravona‐Springer R, Heymann A, et al.~Consumption of ultra‐processed
  food and cognitive decline among older adults with type‐2 diabetes.
  Alzheimer's \& dementia. 2021;17(S10).
\end{enumerate}

\hypertarget{upf-and-bp-1}{%
\subsection{UPF and BP}\label{upf-and-bp-1}}

\begin{enumerate}
\def\labelenumi{\arabic{enumi}.}
\tightlist
\item
  Aceves-Martins M, Link to external site this link will open in a new
  window, Bates RL, Link to external site this link will open in a new
  window, Craig LCA, Chalmers N, et al.~Nutritional Quality,
  Environmental Impact and Cost of Ultra-Processed Foods: A UK
  Food-Based Analysis. International journal of environmental research
  and public health {[}Internet{]}. 2022 {[}cited 2022 Oct 28{]};19(6).
  Available from:
  \url{http://www.proquest.com/publiccontent/docview/2644005015?pq-origsite=primo}
\item
  Aguiar Sarmento R, Peçanha Antonio J, Lamas de Miranda I, Bellicanta
  Nicoletto B, Carnevale de Almeida J. Eating patterns and health
  outcomes in patients with type 2 diabetes. Journal of the Endocrine
  Society. 2018;2(1):42--52.
\item
  Barbosa SS, Sousa LCM, de Oliveira Silva DF, Pimentel JB, Evangelista
  KCM de S, Lyra C de O, et al.~A Systematic Review on
  Processed/Ultra-Processed Foods and Arterial Hypertension in Adults
  and Older People. Nutrients. 2022 Mar 13;14(6):1215.
\item
  Colombet Z, Perignon M, Salanave B, Landais E, Martin-Prevel Y, Allès
  B, et al.~Socioeconomic inequalities in metabolic syndrome in the
  French West Indies. BMC Public Health. 2019 Dec 3;19(1):1620.
\item
  D'Avila HF, Kirsten VR. CONSUMO ENERGÉTICO PROVENIENTE DE ALIMENTOS
  ULTRAPROCESSADOS POR ADOLESCENTES. Revista paulista de pediatria.
  2017;35(1):54--60.
\item
  De Deus Mendonça R, Souza Lopes AC, Pimenta AM, Gea A,
  Martinez-Gonzalez MA, Bes-Rastrollo M. Ultra-processed food
  consumption and the incidence of hypertension in a mediterranean
  cohort: The seguimiento universidad de navarra project. American
  journal of hypertension. 2017;30(4):358--66.
\item
  de Miranda RC, Rauber F, Levy RB. Impact of ultra-processed food
  consumption on metabolic health. Current opinion in lipidology.
  2021;32(1):24--37.
\item
  dos Santos FS, Dias M da S, Mintem GC, de Oliveira IO, Gigante DP.
  Food processing and cardiometabolic risk factors: a systematic review.
  Rev Saude Publica. 54:70.
\item
  Gomez-Smith M, Janik R, Adams C, Lake EM, Thomason LAM, Jeffers MS, et
  al.~Reduced cerebrovascular reactivity and increased resting cerebral
  perfusion in rats exposed to a cafeteria diet. Neuroscience.
  2018;371:166--77.
\item
  Gonçalves VS, Duarte EC, Dutra ES, Barufaldi LA, Carvalho KM.
  Characteristics of the school food environment associated with
  hypertension and obesity in Brazilian adolescents: a multilevel
  analysis of the Study of Cardiovascular Risks in Adolescents (ERICA).
  Public health nutrition. 2019;22(14):2625--34.
\item
  Goodman D, González-Rivas JP, Jaacks LM, Duran M, Marulanda MI, Ugel
  E, et al.~Dietary intake and cardiometabolic risk factors among
  Venezuelan adults: a nationally representative analysis. BMC
  nutrition. 2020;6(1):61--61.
\item
  Ivancovsky‐Wajcman D, Fliss‐Isakov N, Webb M, Bentov I, Shibolet O,
  Kariv R, et al.~Ultra‐processed food is associated with features of
  metabolic syndrome and non‐alcoholic fatty liver disease. Liver
  international. 2021;41(11):2635--45.
\item
  Kityo A, Lee SA. The intake of ultra-processed foods and prevalence of
  chronic kidney disease: The health examinees study. Nutrients.
  2022;14(17):3548-.
\item
  Lee HY. Ultra-processed foods as a less-known risk factor in
  cardiovascular diseases. Korean circulation journal. 2022;52(1):71--3.
\item
  Li M, Link to external site this link will open in a new window, Shi
  Z, Link to external site this link will open in a new window.
  Association between Ultra-Processed Food Consumption and Diabetes in
  Chinese Adults-Results from the China Health and Nutrition Survey.
  Nutrients {[}Internet{]}. 2022 {[}cited 2022 Nov 12{]};14(20).
  Available from:
  \url{https://www.proquest.com/publiccontent/docview/2729520244?parentSessionId=8CgvVWDFcQEhyTTXC\%2B3zh7oBuY1vDlJi2c0\%2Fm7JmQZk\%3D\&pq-origsite=primo\&}
\item
  Li M, Shi Z. Ultra-processed food consumption associated with
  overweight/obesity among Chinese adults---Results from China health
  and nutrition survey 1997--2011. Nutrients. 2021;13(8):2796-.
\item
  Li M, Shi Z. Association between Ultra-Processed Food Consumption and
  Diabetes in Chinese Adults---Results from the China Health and
  Nutrition Survey. Nutrients. 2022 Jan;14(20):4241.
\item
  Lima R, Moreira L, Rossato S, Silva R, Fuchs S. P2-155 Consumption of
  ultra-processed food is associated with blood pressure in hypertensive
  individuals. Journal of epidemiology and community health (1979).
  2011;65(Suppl 1):A263--A263.
\item
  Martínez Steele E, Juul F, Neri D, Rauber F, Monteiro CA. Dietary
  share of ultra-processed foods and metabolic syndrome in the US adult
  population. Preventive medicine. 2019;125:40--8.
\item
  Oliveira T, Ribeiro I, Jurema-Santos G, Nobre I, Santos R, Rodrigues
  C, et al.~Can the consumption of ultra-processed food be associated
  with anthropometric indicators of obesity and blood pressure in
  children 7 to 10 years old? Foods. 2020;9(11):1567-.
\item
  Rauber F, Louzada ML da C, Steele EM, Rezende LFM de, Millett C,
  Monteiro CA, et al.~Ultra-processed foods and excessive free sugar
  intake in the UK: a nationally representative cross-sectional study.
  BMJ Open. 2019 Oct 1;9(10):e027546.
\item
  Rauber F, Steele EM, Louzada ML da C, Millett C, Monteiro CA, Levy RB.
  Ultra-processed food consumption and indicators of obesity in the
  United Kingdom population (2008-2016). Meyre D, editor. PLoS ONE. 2020
  May 1;15(5):e0232676.
\item
  Rezende-Alves K, Hermsdorff HHM, Miranda AE da S, Lopes ACS, Bressan
  J, Pimenta AM. Food processing and risk of hypertension: Cohort of
  universities of minas gerais, brazil (CUME project). Public health
  nutrition. 2021;24(13):4071--9.
\item
  Santos FSD, Dias M da S, Mintem GC, Oliveira IO de, Gigante DP. Food
  processing and cardiometabolic risk factors: a systematic review.
  Revista de saúde pública. 2020;54:70--70.
\item
  Scaranni P de O da S. Ultra-processed foods, changes in blood pressure
  and incidence of hypertension: the Brazilian Longitudinal Study of
  Adult Health (ELSA-Brasil) \textbar{} Public Health Nutrition
  \textbar{} Cambridge Core {[}Internet{]}. {[}cited 2023 Mar 15{]}.
  Available from:
  \url{https://www-cambridge-org.liverpool.idm.oclc.org/core/journals/public-health-nutrition/article/ultraprocessed-foods-changes-in-blood-pressure-and-incidence-of-hypertension-the-brazilian-longitudinal-study-of-adult-health-elsabrasil/1A120EFBE6785C030961E19B94977D9B}
\item
  Scaranni P de O da S, Cardoso L de O, Chor D, Melo ECP, Matos SMA,
  Giatti L, et al.~Ultra-processed foods, changes in blood pressure and
  incidence of hypertension: the Brazilian Longitudinal Study of Adult
  Health (ELSA-Brasil). Public health nutrition. 2021;24(11):3352--60.
\item
  Scaranni P de O da S, Cardoso L de O, Chor D, Melo ECP, Matos SMA,
  Giatti L, et al.~Ultra-processed foods, changes in blood pressure and
  incidence of hypertension: the Brazilian Longitudinal Study of Adult
  Health (ELSA-Brasil). Public Health Nutrition. 2021
  Aug;24(11):3352--60.
\item
  Schulze K. UPF and cardiometabolic health {[}Internet{]}. University
  of Cambridge; 2019 {[}cited 2023 Mar 2{]}. Available from:
  \url{https://www.repository.cam.ac.uk/bitstream/handle/1810/306587/Kai\%20Schulze\%20Thesis\%202020_final.pdf?sequence=1\&isAllowed=y}
\item
  Shim SY, Kim HC, Shim JS. Consumption of ultra-processed food and
  blood pressure in korean adults. Korean circulation journal.
  2022;52(1):60--70.
\item
  Shim SY, Kim HC, Shim JS. Consumption of Ultra-Processed Food and
  Blood Pressure in Korean Adults. Korean Circ J. 2022 Jan;52(1):60--70.
\item
  Smiljanec K, Mbakwe AU, Ramos-Gonzalez M, Mesbah C, Lennon SL.
  Associations of ultra-processed and unprocessed/minimally processed
  food consumption with peripheral and central hemodynamics, and
  arterial stiffness in young healthy adults. Nutrients.
  2020;12(11):1--19.
\item
  Suter PM, Sierro C, Vetter W. Nutritional Factors in the Control of
  Blood Pressure and Hypertension. Nutrition in Clinical Care.
  2002;5(1):9--19.
\item
  Tavares LF, Fonseca SC, Garcia Rosa ML, Yokoo EM. Relationship between
  ultra-processed foods and metabolic syndrome in adolescents from a
  Brazilian Family Doctor Program. Public health nutrition.
  2012;15(1):82--7.
\item
  Tzelefa V, Tsirimiagkou C, Argyris A, Moschonis G, Perogiannakis G,
  Yannakoulia M, et al.~Associations of dietary patterns with blood
  pressure and markers of subclinical arterial damage in adults with
  risk factors for CVD. Public health nutrition. 2021;24(18):6075--84.
\item
  Vilela S, Magalhães V, Severo M, Oliveira A, Torres D, Lopes C. Effect
  of the food processing degree on cardiometabolic health outcomes: A
  prospective approach in childhood. Clinical nutrition (Edinburgh,
  Scotland). 2022;41(10):2235--43.
\item
  Wang M, Du X, Huang W, Xu Y. Ultra-processed foods consumption
  increases the risk of hypertension in adults: A systematic review and
  meta-analysis. American journal of hypertension. 2022;35(10):892--901.
\item
  Wang M, Du X, Huang W, Xu Y. Ultra-Processed Foods Consumption
  Increases the Risk of Hypertension in Adults: A Systematic Review and
  Meta-Analysis. American Journal of Hypertension. 2022 Oct
  1;35(10):892--901.
\end{enumerate}

\hypertarget{bp-upf-and-salt}{%
\subsection{BP UPF and Salt~}\label{bp-upf-and-salt}}

~ What is not known is how UPF cause BP.~ Is it nutrient based? In which
case is this mediated by Salt?~~ Is it other factors? This study looks
only at if Na is part of the causal pathway The thesis is that UPF is
more of a risk than the salt it contains

\begin{enumerate}
\def\labelenumi{\arabic{enumi}.}
\tightlist
\item
  Cappuccio FP, Capewell S. Facts, Issues, and Controversies in Salt
  Reduction for the Prevention of Cardiovascular Disease. 2015;7(1):21.
\item
  Elijovich F, Weinberger MH, Anderson CAM, Appel LJ, Bursztyn M, Cook
  NR, et al.~Salt Sensitivity of Blood Pressure: A Scientific Statement
  From the American Heart Association. Hypertension. 2016
  Sep;68(3):e7--46.
\item
  Elliott P, Stamler J, Nichols R, Dyer AR, Stamler R, Kesteloot H, et
  al.~Intersalt revisited: further analyses of 24 hour sodium excretion
  and blood pressure within and across populations. BMJ. 1996 May
  18;312(7041):1249--53.
\item
  He FJ, MacGregor GA. Reducing Population Salt Intake Worldwide: From
  Evidence to Implementation. Progress in Cardiovascular Diseases. 2010
  Mar 1;52(5):363--82.
\item
  Newman T. High blood pressure: Sodium may not be the culprit
  {[}Internet{]}. Medical News Today. 2017 {[}cited 2022 Oct 14{]}.
  Available from: \url{https://www.medicalnewstoday.com/articles/317099}
\item
  Nilson EAF, Spaniol AM, Santin R da C, Silva SA. Estratégias para
  redução do consumo de nutrientes críticos para a saúde: o caso do
  sódio. Cadernos de saúde pública. 2021;37(suppl 1).
\item
  Sacks FM, Svetkey LP, Vollmer WM, Appel LJ, Bray GA, Harsha D, et
  al.~Effects on Blood Pressure of Reduced Dietary Sodium and the
  Dietary Approaches to Stop Hypertension (DASH) Diet. New England
  Journal of Medicine. 2001 Jan 4;344(1):3--10.
\item
  Vollmer WM, Sacks FM, Ard J, Appel LJ, Bray GA, Simons-Morton DG, et
  al.~Effects of Diet and Sodium Intake on Blood Pressure: Subgroup
  Analysis of the DASH-Sodium Trial. Ann Intern Med. 2001 Dec
  18;135(12):1019.
\item
  Intersalt: an international study of electrolyte excretion and blood
  pressure. Results for 24 hour urinary sodium and potassium excretion.
  Intersalt Cooperative Research Group. BMJ. 1988 Jul
  30;297(6644):319--28.
\item
  Your Guide to Lowering Your Blood Pressure with DASH. US Department of
  Health and Human Services; 1998 p.~64.
\end{enumerate}

\hypertarget{bibliography}{%
\subsection{Bibliography}\label{bibliography}}

\begin{enumerate}
\def\labelenumi{\arabic{enumi}.}
\tightlist
\item
  Kannel WB, Garrison RJ, Dannenberg AL. Secular blood pressure trends
  in normotensive persons: The Framingham Study. Am Heart J. 1993 Apr
  1;125(4):1154--8.
\item
  Kannel WB. Hypertension: Reflections on Risks and Prognostication. Med
  Clin North Am. 2009 May;93(3):541-Contents.
\item
  Bress AP, Cohen JB, Anstey DE, Conroy MB, Ferdinand KC, Fontil V, et
  al.~Inequities in Hypertension Control in the United States Exposed
  and Exacerbated by COVID‐19 and the Role of Home Blood Pressure and
  Virtual Health Care During and After the COVID‐19 Pandemic. J Am Heart
  Assoc. 2021 Jun 1;10(11):e020997.
\item
  Debon R, Bellei EA, Biduski D, Volpi SS, Alves ALS, Portella MR, et
  al.~Effects of using a mobile health application on the health
  conditions of patients with arterial hypertension: A pilot trial in
  the context of Brazil's Family Health Strategy. Sci
  Rep.~2020;10(1):6009--6009.
\item
  Ettehad D, Emdin CA, Kiran A, Anderson SG, Callender T, Emberson J, et
  al.~Blood pressure lowering for prevention of cardiovascular disease
  and death: a systematic review and meta-analysis. The Lancet. 2016 Mar
  5;387(10022):957--67.
\item
  Pringle E, Phillips C, Thijs L, Davidson C, Staessen JA, de Leeuw PW,
  et al.~Systolic blood pressure variability as a risk factor for stroke
  and cardiovascular mortality in the elderly hypertensive population. J
  Hypertens. 2003 Dec;21(12):2251--7.
\item
  Roche M, Onyia I. A quality improvement package for high blood
  pressure (BP) management in general practice, part of a systems
  leadership approach to tackling high BP in Cheshire and Merseyside
  {[}Internet{]}. NICE. NICE; 2018 {[}cited 2022 Jan 19{]}. Available
  from:
  \url{https://www.nice.org.uk/sharedlearning/a-quality-improvement-package-for-high-blood-pressure-bp-management-in-general-practice-part-of-a-systems-leadership-approach-to-tackling-high-bp-in-cheshire-and-merseyside}
\item
  WHO. High blood pressure: a public health problem {[}Internet{]}.
  World Health Organization - Regional Office for the Eastern
  Mediterranean. {[}cited 2022 Jan 19{]}. Available from:
  \url{http://www.emro.who.int/media/world-health-day/public-health-problem-factsheet-2013.html}
\item
  Blood pressure - Action on Salt {[}Internet{]}. {[}cited 2022 Nov
  16{]}. Available from:
  \url{https://www.actiononsalt.org.uk/salthealth/factsheets/pressure/}
\item
  Blood Pressure UK {[}Internet{]}. {[}cited 2022 Jan 27{]}. Available
  from: \url{https://www.bloodpressureuk.org/}
\item
  Boutain DM. Discourses of worry, stress, and high blood pressure in
  rural South Louisiana. J Nurs Scholarsh. 2001 Third
  Quarter;33(3):225--30.
\item
  Colombet Z, Simioni M, Drogue S, Lamani V, Perignon M, Martin-Prevel
  Y, et al.~Demographic and socio-economic shifts partly explain the
  Martinican nutrition transition: an analysis of 10-year health and
  dietary changes (2003--2013) using decomposition models. Public Health
  Nutr. 2021;24(18):6323--34.
\item
  Colombet Z, Schwaller E, Head A, Kypridemos C, Capewell S, O'Flaherty
  M. OP12 Social inequalities in ultra-processed food intakes in the
  United Kingdom: A time trend analysis (2008--2018). J Epidemiol
  Community Health. 2022 Aug 1;76(Suppl 1):A6--7.
\item
  Ji C, Cappuccio FP. Socioeconomic inequality in salt intake in Britain
  10 years after a national salt reduction programme. BMJ Open. 2014 Aug
  26;4(8):e005683--e005683.
\item
  Jones NR, Tong TY, Monsivais P. Meeting UK dietary recommendations is
  associated with higher estimated consumer food costs: an analysis
  using the National Diet and Nutrition Survey and consumer expenditure
  data, 2008--2012. Public Health Nutr. 2018 Apr;21(5):948--56.
\item
  Leeuw E de, Simos J, editors. Healthy cities: the theory, policy, and
  practice of value-based urban planning. New York, NY: Springer; 2017.
  515 p.~
\item
  MacGregor GA, He FJ, Pombo-Rodrigues S. Food and the responsibility
  deal: how the salt reduction strategy was derailed. BMJ. 2015 Apr
  28;350:h1936.
\item
  Institute of Medicine, Food and Nutrition Board, Committee on
  Strategies to Reduce Sodium Intake, Boon CS, Taylor CL, Henney JE.
  Strategies to Reduce Sodium Intake in the United States
  {[}Internet{]}. Washington, D.C., UNITED STATES: National Academies
  Press; 2010 {[}cited 2023 Jan 21{]}. Available from:
  \url{http://ebookcentral.proquest.com/lib/liverpool/detail.action?docID=3378676}
\item
  Laverty AA, Link to external site this link will open in a new window,
  Kypridemos C, Seferidi P, Vamos EP, Pearson-Stuttard J, et
  al.~Quantifying the impact of the Public Health Responsibility Deal on
  salt intake, cardiovascular disease and gastric cancer burdens:
  interrupted time series and microsimulation study. J Epidemiol
  Community Health. 2019 Sep;73(9):881.
\item
  Millett C, Laverty AA, Stylianou N, Bibbins-Domingo K, Pape UJ.
  Impacts of a National Strategy to Reduce Population Salt Intake in
  England: Serial Cross Sectional Study. PLoS ONE. 2012 Jan
  4;7(1):e29836.
\item
  Moreira PVL, Baraldi LG, Moubarac JC, Monteiro CA, Newton A, Capewell
  S, et al.~Comparing Different Policy Scenarios to Reduce the
  Consumption of Ultra-Processed Foods in UK: Impact on Cardiovascular
  Disease Mortality Using a Modelling Approach. Hernandez AV, editor.
  PLOS ONE. 2015 Feb 13;10(2):e0118353.
\item
  National Food Strategy, editor. National Food Strategy\,:\,: part one.
  {[}Internet{]}. London\,: National Food Strategy,; 2020. Available
  from: \url{https://www.nationalfoodstrategy.org/partone/}
\item
  Cappuccio FP, Capewell S. Facts, Issues, and Controversies in Salt
  Reduction for the Prevention of Cardiovascular Disease. 2015;7(1):21.
\item
  Intersalt: an international study of electrolyte excretion and blood
  pressure. Results for 24 hour urinary sodium and potassium excretion.
  Intersalt Cooperative Research Group. BMJ. 1988 Jul
  30;297(6644):319--28.
\item
  Elliott P, Stamler J, Nichols R, Dyer AR, Stamler R, Kesteloot H, et
  al.~Intersalt revisited: further analyses of 24 hour sodium excretion
  and blood pressure within and across populations. BMJ. 1996 May
  18;312(7041):1249--53.
\item
  Newman T. High blood pressure: Sodium may not be the culprit
  {[}Internet{]}. Medical News Today. 2017 {[}cited 2022 Oct 14{]}.
  Available from: \url{https://www.medicalnewstoday.com/articles/317099}
\item
  Elijovich F, Weinberger MH, Anderson CAM, Appel LJ, Bursztyn M, Cook
  NR, et al.~Salt Sensitivity of Blood Pressure: A Scientific Statement
  From the American Heart Association. Hypertens Dallas Tex 1979. 2016
  Sep;68(3):e7--46.
\item
  Your Guide to Lowering Your Blood Pressure with DASH. US Department of
  Health and Human Services; 1998 p.~64.
\item
  Reports Outline Obesity, Fitness and Wellness Findings from Federal
  University Vicosa (Effects of Minimally and Ultra-processed Foods On
  Blood Pressure In Brazilian Adults: a Two-year Follow Up of the Cume
  Project). Obes Fit Wellness Week. 2023;3265-.
\item
  Vollmer WM, Sacks FM, Ard J, Appel LJ, Bray GA, Simons-Morton DG, et
  al.~Effects of Diet and Sodium Intake on Blood Pressure: Subgroup
  Analysis of the DASH-Sodium Trial. Ann Intern Med. 2001 Dec
  18;135(12):1019.
\item
  Sacks FM, Svetkey LP, Vollmer WM, Appel LJ, Bray GA, Harsha D, et
  al.~Effects on Blood Pressure of Reduced Dietary Sodium and the
  Dietary Approaches to Stop Hypertension (DASH) Diet. N Engl J Med.
  2001 Jan 4;344(1):3--10.
\item
  Nilson EAF, Spaniol AM, Santin R da C, Silva SA. Estratégias para
  redução do consumo de nutrientes críticos para a saúde: o caso do
  sódio. Cad Saúde Pública. 2021;37(suppl 1).
\item
  He FJ, MacGregor GA. Reducing Population Salt Intake Worldwide: From
  Evidence to Implementation. Prog Cardiovasc Dis. 2010 Mar
  1;52(5):363--82.
\item
  Cuj M, Grabinsky L, Yates-Doerr E. Cultures of Nutrition:
  Classification, Food Policy, and Health. Med Anthropol. 2021 Jan
  2;40(1):79--97.
\item
  Dickie S, Woods J, Machado P, Lawrence M. Nutrition Classification
  Schemes for Informing Nutrition Policy in Australia: Nutrient-Based,
  Food-Based, or Dietary-Based? Curr Dev Nutr. 2022 Jul 4;6(8):nzac112.
\item
  Romero Ferreiro C, Lora Pablos D, Gómez de la Cámara A. Two Dimensions
  of Nutritional Value: Nutri-Score and NOVA. Nutrients. 2021 Aug
  13;13(8):2783.
\item
  A. A, Gan HJ, M.Y. H, K. KS, Zainudin AA. Food classification system
  based on food processing and its relationship with nutritional status
  of adults in Terengganu, Malaysia. Food Res. 2019;4(2):539--46.
\item
  Bourdieu P, Bourdieu P. Distinction: a social critique of the
  judgement of taste. 11. print. Cambridge, Mass: Harvard Univ. Press;
  2002. 613 p.~
\item
  A Bourdieu'dian Analysis for the Construction of an Education in Tea
  {[}Internet{]}. Tea Technique. 2021 {[}cited 2022 Apr 30{]}. Available
  from:
  \url{https://www.teatechnique.org/a-bourdieudian-analysis-for-the-construction-of-an-education-in-tea/}
\item
  Dickie S, Woods J, Machado P, Lawrence M. A novel food
  processing-based nutrition classification scheme for guiding policy
  actions applied to the Australian food supply. Front Nutr
  {[}Internet{]}. 2023 {[}cited 2023 Feb 10{]};10. Available from:
  \url{https://www.frontiersin.org/articles/10.3389/fnut.2023.1071356}
\item
  Monteiro CA. Nutrition and health. The issue is not food, nor
  nutrients, so much as processing. Public Health Nutr. 2009
  May;12(5):729--31.
\item
  Monteiro CA, Cannon G, Levy R, Moubarac JC, Jaime P, Martins AP, et
  al.~NOVA. The star shines bright. World Nutr. 2016 Jan
  7;7(1--3):28--38.
\item
  Monteiro CA, Levy RB, Claro RM, Castro IRR de, Cannon G. A new
  classification of foods based on the extent and purpose of their
  processing. Cad Saúde Pública. 2010 Nov;26:2039--49.
\item
  Monteiro CA, Levy RB, Claro RM, Castro IRR de, Cannon G. Uma nova
  classificação de alimentos baseada na extensão e propósito do seu
  processamento. Cad Saúde Pública. 2010 Nov;26:2039--49.
\item
  Monteiro CA, Moubarac JC, Cannon G, Ng SW, Popkin B. Ultra-processed
  products are becoming dominant in the global food system. Obes
  Rev.~2013;14(S2):21--8.
\item
  Monteiro CA, Astrup A. Does the concept of ``ultra-processed foods''
  help inform dietary guidelines, beyond conventional classification
  systems? YES. Am J Clin Nutr. 2022 Dec 1;116(6):1476--81.
\item
  Astrup A, Monteiro CA. Does the concept of ``ultra-processed foods''
  help inform dietary guidelines, beyond conventional classification
  systems? Debate consensus. Am J Clin Nutr. 2022 Dec 1;116(6):1489--91.
\item
  Aceves-Martins M, Bates RL, Craig LCA, Chalmers N, Horgan G, Boskamp
  B, et al.~Nutritional Quality, Environmental Impact and Cost of
  Ultra-Processed Foods: A UK Food-Based Analysis. Int J Environ Res
  Public Health. 2022 Mar 8;19(6):3191.
\item
  Rauber F, Louzada ML da C, Steele EM, Rezende LFM de, Millett C,
  Monteiro CA, et al.~Ultra-processed foods and excessive free sugar
  intake in the UK: a nationally representative cross-sectional study.
  BMJ Open. 2019 Oct 1;9(10):e027546.
\item
  Wang L, Martínez Steele E, Du M, Pomeranz JL, O'Connor LE, Herrick KA,
  et al.~Trends in Consumption of Ultraprocessed Foods Among US Youths
  Aged 2-19 Years, 1999-2018. JAMA. 2021 Aug 10;326(6):519--30. 1.
  Kannel WB, Garrison RJ, Dannenberg AL. Secular blood pressure trends
  in normotensive persons: The Framingham Study. Am Heart J. 1993 Apr
  1;125(4):1154--8.
\end{enumerate}

\newpage

\hypertarget{method}{%
\section{Method}\label{method}}

\hypertarget{introduction-2}{%
\subsection{Introduction}\label{introduction-2}}

This section takes the research question and explains how the data is
used to answer the question.

There will be a description of the study and data collection. Then a
section on governance and ethics in this project.

Data analysis starts with the relevant variables being identified and
extracted. Some data may need to be recalculated or to be processed to
make a more useable form. The population will be reviewed. Then there is
consideration of groups to be excluded.

There is a description of the data. The second analysis section compares
the data across the annual cohorts. The next analysis section involves
using linear regression to identify correlations. Firstly, between the
BP and each of the key variables. Then between other pairs of key
variables.

Multivariable regression models are then generated. These models are
examined to identify the relative importance of the different variables
in developing an optimal model and what these models tell us about the
relationship between our variables. A summary and conclusion will bring
all these together.

\hypertarget{research-question}{%
\subsection{Research Question}\label{research-question}}

What proportion of the association between blood pressure (SBP) and UPF
intake can be explained by the changes in salt intake in England between
2008 and 2019?

The question can be split into parts, What was intake of salt between
2008 and 2019? What was intake of UPF between 2008 and 2019? What was BP
between 2008 and 2019? Did each of these change over that time and how?
Did the changes in any one affect any other? What are the sizes of the
changes? Which element was most important in these changes?

All of these questions look for numbers as answers.

Answering the question starts with collecting a sample of participants.
Measurements are taken, and then collated. The collected numbers are
then compared in different ways to answer each part of the question.

\hypertarget{national-dietary-and-nutritional-survey}{%
\subsection{National Dietary and Nutritional
Survey}\label{national-dietary-and-nutritional-survey}}

This survey is a collaboration between government departments
responsible for health and for food production. They have engaged
academic partners to deliver reports on diet and nutrition across the
United Kingdom. The study is designed to be representative across the
whole area.

\hypertarget{study-design}{%
\subsubsection{Study design}\label{study-design}}

This is a rolling cohort study which each year selects a new cohort of
participants. The sample is approximately 1000 per year with 50\%
adults. The design has a random selection across postal units (psu).
This is stratified to ensure a representative sample across the four
nations and across regions within those countries. The sample is also
representative for age and sex.

Having taken up the study, participants complete a 4 day food diary, and
have an interview with a nurse which includes taking several
measurements. Weighting is given for each annual survey to enable
comparison across the years taking account for alterations in uptake and
response completion.

\hypertarget{ndns-dataset}{%
\subsubsection{NDNS Dataset}\label{ndns-dataset}}

The data from the NDNS study contains items about each individual,and
their household. It contains a table with each item of food as recorded
in their diary. There is a table with the overall intake of each of a
large range of nutrients for the whole period. This is calculated from
the diary using nutritional tables which are published as part of the
dataset. The dataset is available via the UK national Data service for
research purposes.

NDNS began before Monteiro's processing based classification, Nova , was
developed. There is no record of Nova food type in NDNS. This has been
calculated from the food descriptions. I have used a table from Rauber
et al.~for Nova values in NDNS.

\hypertarget{university-research-governance-and-ethical-review}{%
\subsubsection{University Research Governance and Ethical
Review}\label{university-research-governance-and-ethical-review}}

The research has been carried out under the University governance. A
proposal was discussed and agreed within the public health department.
The need for ethical review was considered using the university research
tool. The fact that the data is anonymised and there was no contact with
participants means that there is minimal risk of harm to research
participants. A certificate from the ethics department is in the
appendix.

Other ethical issues include data custodianship ensuring that the the
rights of the owners of the data and of the participants are still
considered as part of the process of analysis and dissemination of the
research.

Issues around the power structures which lead to privilege one research
project or proposal over another are considered more in the
positionality section.

\hypertarget{data-processing}{%
\subsubsection{Data Processing}\label{data-processing}}

The storage of the data is in keeping with the research governance
agreements of the University and the Data set owners. The data is read
from its files using `r-studio' with the processing being carried out
using packages available from CRAN. I have used files which had been
amalgamated into four batches. These are 2008-2012, 2013-2014,
2015-2016, 2017-2019.

Once the data labels are made consistent across the batches, weighting
recalculation is done. This generates values which account for
differences in population balance across the annual cohorts. These
result from differences in compliance and uptake within and across the
years.

The years are amalgamated and the nature of the variables is specified.

\hypertarget{exclusions}{%
\subsubsection{Exclusions}\label{exclusions}}

The relationship between salt and systolic blood pressure may be
different in individuals with pathologically high BP. Those taking BP
controlling medications may have a different relationship to sodium and
UPF. These patients were excluded from the main analysis, however this
affected the sample size and skewed the male female ratio. Analysis was
done with exclusion and this produced results in line with those
presented, but of smaller magnitude. This additional analysis is not
presented here.

\hypertarget{description-of-the-data}{%
\subsection{Description of the data}\label{description-of-the-data}}

The data is summarised for the key continuous variables. The key
variables are systolic BP (omsysval), UPF intake (Epcnt\_4) and Sodium
intake (sodiummg). These variables are the ones which most relate to the
research question. Table x shows the data which has been balanced using
the weightings provided by the NDNS research team.

There are a number of related variables in the dataset. These were
chosen for relevance, reliability and practicality. These variables are
ones which can also influence BP. They include Age, Sex, BMI, height and
weight. Age at completion of education (educfinh), and IMD are also
used.

The omsysval is a validated measurement with significant quality
assessment within the dataset. Raw systolic BP values are present in the
dataset but are made up of data with issues around quality. In
particular the systolic BP values are assessed for the effects of
exercise, temperature and ill health. The variable omsysval is a quality
assured mean value which is reliable across the dataset.

The sodium value is one calculated from intake based on food diaries and
standard food nutrient values. This only reflects standard foods and is
the result of assumptions about the content being consistent. Serum
sodium values are available for the early dataset, but not the later
one. There are also values for 24 urinary sodium which is probably a
better indicator of dietary sodium for parts of the dataset, but again
these are not found in both time periods. Though they were part of a
supplementary study.

The food diaries need processing to identify the UPF intake. Each
persons food diary entries are assessed against the Nova food
classification from Rauber. Then the weight and energy content of the
days food is calculated by Nova group. This is added to the intake for
the other 3 days and the total intake by Nova group established.

The percentage of the total intake of energy (Epcnt\_4) is then
calculated for each of the 4 Nova categories. Nova group 4 or UPF intake
is used for the study.

Mean values for the data are displayed with a comparison for weighted
values. The exposure variables are sodium intake (Sodiummg), and ultra
processed food intake (Epcnt\_4). The outcome variable, the mean
systolic blood pressure (omsysval).

\hypertarget{analysis-of-change-over-survey-years}{%
\subsubsection{Analysis of Change over Survey
Years}\label{analysis-of-change-over-survey-years}}

The second phase of analysis shows how the key variables have changed
over the survey years cohorts. This will show separately how the inputs
and outputs have changed.

These are not the same participants so matched analysis, or time series
analysis is not directly applicable.

Plots will be given to show the values in each of the available cohorts.

Other variables in the data are compared across to assess how the data
changes. Statistical significance of changes in the data are shown by
p.values with continuous data, and categorical data analysed using chi
squared tables.

\hypertarget{univariable-regression-of-key-variables}{%
\subsubsection{Univariable Regression of key
variables}\label{univariable-regression-of-key-variables}}

Analysis of the correlation between BP and sodium intake, and then BP
and UPF intake is done using linear regression. This will give an
indicator of the direction, and strength of any relationship between the
variables. There is also anova analysis to understand the statistical
significance of these results. Comparison is also made with Age, and
between each of the variables. This will show where significant
relationships are present.

\hypertarget{multiple-regression-on-systolic-bp-age-epcnt_4}{%
\subsubsection{Multiple Regression on Systolic BP (?age,
?Epcnt\_4)}\label{multiple-regression-on-systolic-bp-age-epcnt_4}}

Multivariable regression models are then developed to understand the
interactions between variables and to develop a mathematical model of
the relationship. The optimal model is one which best explains the
pattern of data, but which also makes practical sense for the wider
understanding of relationships. Assessment techniques try to understand
the importance of including particular variables, and the form in which
they are best included. Anova analysis here identifies how the addition
of different variables changes the significance of other variables. This
can suggest causative relationships. The resultant p.values help to
establish the statistical significance of the results.

\hypertarget{aic-and-sensitivty-anaylsis}{%
\subsubsection{AIC and sensitivty
Anaylsis}\label{aic-and-sensitivty-anaylsis}}

This section compares models side by side using assessment techniques to
identify the best way of describing the data. The `best' in part is
determined by the whether a model is needed to predict more data, or
just to understand the data available. Here it is about how best to
describe the relationship between Na, UPF, and BP.

\hypertarget{method-conclusion}{%
\subsection{Method Conclusion}\label{method-conclusion}}

This section has highlighted how the material for the study is brought
together and how the governance and ethics fit with the data collection,
processing and analysis to help us to derive the results which will be
presented in the next section.

\newpage

\hypertarget{results}{%
\section{Results}\label{results}}

\hypertarget{results-introduction}{%
\subsection{Results Introduction}\label{results-introduction}}

The results derive from the method outlined above and follow the pattern
described. I will discuss the results having already described the
method.

\hypertarget{description-of-the-data-1}{%
\subsection{Description of the Data}\label{description-of-the-data-1}}

This first table highlights the key variables from the years 2008-2019.
These are weighted values And see Table @ref(tab:keydata)

\global\setlength{\Oldarrayrulewidth}{\arrayrulewidth}

\global\setlength{\Oldtabcolsep}{\tabcolsep}

\setlength{\tabcolsep}{0pt}

\renewcommand*{\arraystretch}{1.5}



\providecommand{\ascline}[3]{\noalign{\global\arrayrulewidth #1}\arrayrulecolor[HTML]{#2}\cline{#3}}

\begin{longtable}[c]{|p{2.86in}|p{1.27in}|p{1.27in}|p{1.27in}|p{1.27in}|p{1.27in}|p{1.27in}|p{1.27in}|p{1.27in}|p{1.27in}|p{1.37in}|p{1.37in}}

\caption{NDNS\ year\ 1-11\ data\ over\ years}\\

\ascline{1pt}{000000}{1-12}

\multicolumn{1}{>{\raggedright}p{\dimexpr 2.86in+0\tabcolsep}}{\textcolor[HTML]{000000}{\fontsize{11}{11}\selectfont{\textbf{Characteristic}}}} & \multicolumn{1}{>{\centering}p{\dimexpr 1.27in+0\tabcolsep}}{\textcolor[HTML]{000000}{\fontsize{11}{11}\selectfont{\textbf{1}}}\textcolor[HTML]{000000}{\fontsize{11}{11}\selectfont{,\ N\ =\ 1,459}}\textcolor[HTML]{000000}{\textsuperscript{\fontsize{11}{11}\selectfont{1}}}} & \multicolumn{1}{>{\centering}p{\dimexpr 1.27in+0\tabcolsep}}{\textcolor[HTML]{000000}{\fontsize{11}{11}\selectfont{\textbf{2}}}\textcolor[HTML]{000000}{\fontsize{11}{11}\selectfont{,\ N\ =\ 1,429}}\textcolor[HTML]{000000}{\textsuperscript{\fontsize{11}{11}\selectfont{1}}}} & \multicolumn{1}{>{\centering}p{\dimexpr 1.27in+0\tabcolsep}}{\textcolor[HTML]{000000}{\fontsize{11}{11}\selectfont{\textbf{3}}}\textcolor[HTML]{000000}{\fontsize{11}{11}\selectfont{,\ N\ =\ 1,372}}\textcolor[HTML]{000000}{\textsuperscript{\fontsize{11}{11}\selectfont{1}}}} & \multicolumn{1}{>{\centering}p{\dimexpr 1.27in+0\tabcolsep}}{\textcolor[HTML]{000000}{\fontsize{11}{11}\selectfont{\textbf{4}}}\textcolor[HTML]{000000}{\fontsize{11}{11}\selectfont{,\ N\ =\ 1,432}}\textcolor[HTML]{000000}{\textsuperscript{\fontsize{11}{11}\selectfont{1}}}} & \multicolumn{1}{>{\centering}p{\dimexpr 1.27in+0\tabcolsep}}{\textcolor[HTML]{000000}{\fontsize{11}{11}\selectfont{\textbf{5}}}\textcolor[HTML]{000000}{\fontsize{11}{11}\selectfont{,\ N\ =\ 1,485}}\textcolor[HTML]{000000}{\textsuperscript{\fontsize{11}{11}\selectfont{1}}}} & \multicolumn{1}{>{\centering}p{\dimexpr 1.27in+0\tabcolsep}}{\textcolor[HTML]{000000}{\fontsize{11}{11}\selectfont{\textbf{6}}}\textcolor[HTML]{000000}{\fontsize{11}{11}\selectfont{,\ N\ =\ 1,362}}\textcolor[HTML]{000000}{\textsuperscript{\fontsize{11}{11}\selectfont{1}}}} & \multicolumn{1}{>{\centering}p{\dimexpr 1.27in+0\tabcolsep}}{\textcolor[HTML]{000000}{\fontsize{11}{11}\selectfont{\textbf{7}}}\textcolor[HTML]{000000}{\fontsize{11}{11}\selectfont{,\ N\ =\ 1,442}}\textcolor[HTML]{000000}{\textsuperscript{\fontsize{11}{11}\selectfont{1}}}} & \multicolumn{1}{>{\centering}p{\dimexpr 1.27in+0\tabcolsep}}{\textcolor[HTML]{000000}{\fontsize{11}{11}\selectfont{\textbf{8}}}\textcolor[HTML]{000000}{\fontsize{11}{11}\selectfont{,\ N\ =\ 1,405}}\textcolor[HTML]{000000}{\textsuperscript{\fontsize{11}{11}\selectfont{1}}}} & \multicolumn{1}{>{\centering}p{\dimexpr 1.27in+0\tabcolsep}}{\textcolor[HTML]{000000}{\fontsize{11}{11}\selectfont{\textbf{9}}}\textcolor[HTML]{000000}{\fontsize{11}{11}\selectfont{,\ N\ =\ 1,444}}\textcolor[HTML]{000000}{\textsuperscript{\fontsize{11}{11}\selectfont{1}}}} & \multicolumn{1}{>{\centering}p{\dimexpr 1.37in+0\tabcolsep}}{\textcolor[HTML]{000000}{\fontsize{11}{11}\selectfont{\textbf{10}}}\textcolor[HTML]{000000}{\fontsize{11}{11}\selectfont{,\ N\ =\ 1,481}}\textcolor[HTML]{000000}{\textsuperscript{\fontsize{11}{11}\selectfont{1}}}} & \multicolumn{1}{>{\centering}p{\dimexpr 1.37in+0\tabcolsep}}{\textcolor[HTML]{000000}{\fontsize{11}{11}\selectfont{\textbf{11}}}\textcolor[HTML]{000000}{\fontsize{11}{11}\selectfont{,\ N\ =\ 1,345}}\textcolor[HTML]{000000}{\textsuperscript{\fontsize{11}{11}\selectfont{1}}}} \\

\ascline{1pt}{000000}{1-12}\endhead



\multicolumn{12}{>{\raggedright}p{\dimexpr 16.99in+22\tabcolsep}}{\textcolor[HTML]{000000}{\textsuperscript{\fontsize{11}{11}\selectfont{1}}}\textcolor[HTML]{000000}{\fontsize{11}{11}\selectfont{Mean\ (SD)}}} \\

\endfoot



\multicolumn{1}{>{\raggedright}p{\dimexpr 2.86in+0\tabcolsep}}{\textcolor[HTML]{000000}{\fontsize{11}{11}\selectfont{Sodium\ (mg)\ diet\ only}}} & \multicolumn{1}{>{\centering}p{\dimexpr 1.27in+0\tabcolsep}}{\textcolor[HTML]{000000}{\fontsize{11}{11}\selectfont{2,257\ (878)}}} & \multicolumn{1}{>{\centering}p{\dimexpr 1.27in+0\tabcolsep}}{\textcolor[HTML]{000000}{\fontsize{11}{11}\selectfont{2,208\ (827)}}} & \multicolumn{1}{>{\centering}p{\dimexpr 1.27in+0\tabcolsep}}{\textcolor[HTML]{000000}{\fontsize{11}{11}\selectfont{2,184\ (830)}}} & \multicolumn{1}{>{\centering}p{\dimexpr 1.27in+0\tabcolsep}}{\textcolor[HTML]{000000}{\fontsize{11}{11}\selectfont{2,077\ (799)}}} & \multicolumn{1}{>{\centering}p{\dimexpr 1.27in+0\tabcolsep}}{\textcolor[HTML]{000000}{\fontsize{11}{11}\selectfont{2,010\ (742)}}} & \multicolumn{1}{>{\centering}p{\dimexpr 1.27in+0\tabcolsep}}{\textcolor[HTML]{000000}{\fontsize{11}{11}\selectfont{1,988\ (765)}}} & \multicolumn{1}{>{\centering}p{\dimexpr 1.27in+0\tabcolsep}}{\textcolor[HTML]{000000}{\fontsize{11}{11}\selectfont{1,987\ (798)}}} & \multicolumn{1}{>{\centering}p{\dimexpr 1.27in+0\tabcolsep}}{\textcolor[HTML]{000000}{\fontsize{11}{11}\selectfont{1,945\ (822)}}} & \multicolumn{1}{>{\centering}p{\dimexpr 1.27in+0\tabcolsep}}{\textcolor[HTML]{000000}{\fontsize{11}{11}\selectfont{1,924\ (775)}}} & \multicolumn{1}{>{\centering}p{\dimexpr 1.37in+0\tabcolsep}}{\textcolor[HTML]{000000}{\fontsize{11}{11}\selectfont{1,892\ (724)}}} & \multicolumn{1}{>{\centering}p{\dimexpr 1.37in+0\tabcolsep}}{\textcolor[HTML]{000000}{\fontsize{11}{11}\selectfont{1,929\ (762)}}} \\





\multicolumn{1}{>{\raggedright}p{\dimexpr 2.86in+0\tabcolsep}}{\textcolor[HTML]{000000}{\fontsize{11}{11}\selectfont{Epcnt\_4}}} & \multicolumn{1}{>{\centering}p{\dimexpr 1.27in+0\tabcolsep}}{\textcolor[HTML]{000000}{\fontsize{11}{11}\selectfont{49\ (14)}}} & \multicolumn{1}{>{\centering}p{\dimexpr 1.27in+0\tabcolsep}}{\textcolor[HTML]{000000}{\fontsize{11}{11}\selectfont{50\ (15)}}} & \multicolumn{1}{>{\centering}p{\dimexpr 1.27in+0\tabcolsep}}{\textcolor[HTML]{000000}{\fontsize{11}{11}\selectfont{49\ (15)}}} & \multicolumn{1}{>{\centering}p{\dimexpr 1.27in+0\tabcolsep}}{\textcolor[HTML]{000000}{\fontsize{11}{11}\selectfont{49\ (15)}}} & \multicolumn{1}{>{\centering}p{\dimexpr 1.27in+0\tabcolsep}}{\textcolor[HTML]{000000}{\fontsize{11}{11}\selectfont{48\ (15)}}} & \multicolumn{1}{>{\centering}p{\dimexpr 1.27in+0\tabcolsep}}{\textcolor[HTML]{000000}{\fontsize{11}{11}\selectfont{50\ (16)}}} & \multicolumn{1}{>{\centering}p{\dimexpr 1.27in+0\tabcolsep}}{\textcolor[HTML]{000000}{\fontsize{11}{11}\selectfont{47\ (15)}}} & \multicolumn{1}{>{\centering}p{\dimexpr 1.27in+0\tabcolsep}}{\textcolor[HTML]{000000}{\fontsize{11}{11}\selectfont{45\ (16)}}} & \multicolumn{1}{>{\centering}p{\dimexpr 1.27in+0\tabcolsep}}{\textcolor[HTML]{000000}{\fontsize{11}{11}\selectfont{45\ (16)}}} & \multicolumn{1}{>{\centering}p{\dimexpr 1.37in+0\tabcolsep}}{\textcolor[HTML]{000000}{\fontsize{11}{11}\selectfont{45\ (15)}}} & \multicolumn{1}{>{\centering}p{\dimexpr 1.37in+0\tabcolsep}}{\textcolor[HTML]{000000}{\fontsize{11}{11}\selectfont{47\ (16)}}} \\





\multicolumn{1}{>{\raggedright}p{\dimexpr 2.86in+0\tabcolsep}}{\textcolor[HTML]{000000}{\fontsize{11}{11}\selectfont{(D)\ Omron\ valid\ mean\ systolic\ BP}}} & \multicolumn{1}{>{\centering}p{\dimexpr 1.27in+0\tabcolsep}}{\textcolor[HTML]{000000}{\fontsize{11}{11}\selectfont{125\ (19)}}} & \multicolumn{1}{>{\centering}p{\dimexpr 1.27in+0\tabcolsep}}{\textcolor[HTML]{000000}{\fontsize{11}{11}\selectfont{124\ (16)}}} & \multicolumn{1}{>{\centering}p{\dimexpr 1.27in+0\tabcolsep}}{\textcolor[HTML]{000000}{\fontsize{11}{11}\selectfont{124\ (18)}}} & \multicolumn{1}{>{\centering}p{\dimexpr 1.27in+0\tabcolsep}}{\textcolor[HTML]{000000}{\fontsize{11}{11}\selectfont{124\ (16)}}} & \multicolumn{1}{>{\centering}p{\dimexpr 1.27in+0\tabcolsep}}{\textcolor[HTML]{000000}{\fontsize{11}{11}\selectfont{122\ (17)}}} & \multicolumn{1}{>{\centering}p{\dimexpr 1.27in+0\tabcolsep}}{\textcolor[HTML]{000000}{\fontsize{11}{11}\selectfont{120\ (18)}}} & \multicolumn{1}{>{\centering}p{\dimexpr 1.27in+0\tabcolsep}}{\textcolor[HTML]{000000}{\fontsize{11}{11}\selectfont{124\ (19)}}} & \multicolumn{1}{>{\centering}p{\dimexpr 1.27in+0\tabcolsep}}{\textcolor[HTML]{000000}{\fontsize{11}{11}\selectfont{121\ (18)}}} & \multicolumn{1}{>{\centering}p{\dimexpr 1.27in+0\tabcolsep}}{\textcolor[HTML]{000000}{\fontsize{11}{11}\selectfont{121\ (17)}}} & \multicolumn{1}{>{\centering}p{\dimexpr 1.37in+0\tabcolsep}}{\textcolor[HTML]{000000}{\fontsize{11}{11}\selectfont{122\ (16)}}} & \multicolumn{1}{>{\centering}p{\dimexpr 1.37in+0\tabcolsep}}{\textcolor[HTML]{000000}{\fontsize{11}{11}\selectfont{0\ (0)}}} \\





\multicolumn{1}{>{\raggedright}p{\dimexpr 2.86in+0\tabcolsep}}{\textcolor[HTML]{000000}{\fontsize{11}{11}\selectfont{Unknown}}} & \multicolumn{1}{>{\centering}p{\dimexpr 1.27in+0\tabcolsep}}{\textcolor[HTML]{000000}{\fontsize{11}{11}\selectfont{609}}} & \multicolumn{1}{>{\centering}p{\dimexpr 1.27in+0\tabcolsep}}{\textcolor[HTML]{000000}{\fontsize{11}{11}\selectfont{639}}} & \multicolumn{1}{>{\centering}p{\dimexpr 1.27in+0\tabcolsep}}{\textcolor[HTML]{000000}{\fontsize{11}{11}\selectfont{604}}} & \multicolumn{1}{>{\centering}p{\dimexpr 1.27in+0\tabcolsep}}{\textcolor[HTML]{000000}{\fontsize{11}{11}\selectfont{654}}} & \multicolumn{1}{>{\centering}p{\dimexpr 1.27in+0\tabcolsep}}{\textcolor[HTML]{000000}{\fontsize{11}{11}\selectfont{551}}} & \multicolumn{1}{>{\centering}p{\dimexpr 1.27in+0\tabcolsep}}{\textcolor[HTML]{000000}{\fontsize{11}{11}\selectfont{574}}} & \multicolumn{1}{>{\centering}p{\dimexpr 1.27in+0\tabcolsep}}{\textcolor[HTML]{000000}{\fontsize{11}{11}\selectfont{588}}} & \multicolumn{1}{>{\centering}p{\dimexpr 1.27in+0\tabcolsep}}{\textcolor[HTML]{000000}{\fontsize{11}{11}\selectfont{541}}} & \multicolumn{1}{>{\centering}p{\dimexpr 1.27in+0\tabcolsep}}{\textcolor[HTML]{000000}{\fontsize{11}{11}\selectfont{562}}} & \multicolumn{1}{>{\centering}p{\dimexpr 1.37in+0\tabcolsep}}{\textcolor[HTML]{000000}{\fontsize{11}{11}\selectfont{529}}} & \multicolumn{1}{>{\centering}p{\dimexpr 1.37in+0\tabcolsep}}{\textcolor[HTML]{000000}{\fontsize{11}{11}\selectfont{1,345}}} \\

\ascline{1pt}{000000}{1-12}



\end{longtable}



\arrayrulecolor[HTML]{000000}

\global\setlength{\arrayrulewidth}{\Oldarrayrulewidth}

\global\setlength{\tabcolsep}{\Oldtabcolsep}

\renewcommand*{\arraystretch}{1}

This tables shows the change between annual cohorts.

These plots show how the percentage of energy derived from UPF, the
sodium intake, and the Systolic BP have changed over the years. The
graphs show that there is not a clear visible difference between the
years. Statistical analysis will follow. These next box plots show the
difference between the sexes in the key variables.

\hypertarget{analysis-of-change-over-survey-years-1}{%
\subsection{Analysis of Change over Survey
Years}\label{analysis-of-change-over-survey-years-1}}

comparing UPF and Sodium intake calculated from diet

The sodium levels are compared across the survey years and show a
statistically significant trend.

The pcnt UPF intake in over the same period shows a similar trend.

It seems the mean percentage UPF intake changes from 48.8\% to 59.2\%
energy and this increase is statistically significant. The mean sodium
intake has changed from 2156.30 mg to 2574.33 mg and is also
statistically significant with a p value less than 0.05.

what about outcome BP?

The next t tests compare mean systolic values in the two time periods
and then the mean diastolic values.

\begin{verbatim}
## Warning: fonts used in `flextable` are ignored because the `pdflatex` engine is
## used and not `xelatex` or `lualatex`. You can avoid this warning by using the
## `set_flextable_defaults(fonts_ignore=TRUE)` command or use a compatible engine
## by defining `latex_engine: xelatex` in the YAML header of the R Markdown
## document.
\end{verbatim}

\global\setlength{\Oldarrayrulewidth}{\arrayrulewidth}

\global\setlength{\Oldtabcolsep}{\tabcolsep}

\setlength{\tabcolsep}{0pt}

\renewcommand*{\arraystretch}{1.5}



\providecommand{\ascline}[3]{\noalign{\global\arrayrulewidth #1}\arrayrulecolor[HTML]{#2}\cline{#3}}

\begin{longtable}[c]{|p{1.80in}|p{1.70in}|p{0.68in}|p{1.17in}|p{0.93in}}



\ascline{1pt}{000000}{1-5}

\multicolumn{1}{>{\raggedright}p{\dimexpr 1.8in+0\tabcolsep}}{\textcolor[HTML]{000000}{\fontsize{11}{11}\selectfont{\textbf{Group}}}} & \multicolumn{1}{>{\raggedright}p{\dimexpr 1.7in+0\tabcolsep}}{\textcolor[HTML]{000000}{\fontsize{11}{11}\selectfont{\textbf{Characteristic}}}} & \multicolumn{1}{>{\centering}p{\dimexpr 0.68in+0\tabcolsep}}{\textcolor[HTML]{000000}{\fontsize{11}{11}\selectfont{\textbf{Beta}}}} & \multicolumn{1}{>{\centering}p{\dimexpr 1.17in+0\tabcolsep}}{\textcolor[HTML]{000000}{\fontsize{11}{11}\selectfont{\textbf{95\%\ CI}}}\textcolor[HTML]{000000}{\textsuperscript{\fontsize{11}{11}\selectfont{1}}}} & \multicolumn{1}{>{\centering}p{\dimexpr 0.93in+0\tabcolsep}}{\textcolor[HTML]{000000}{\fontsize{11}{11}\selectfont{\textbf{p-value}}}} \\

\ascline{1pt}{000000}{1-5}\endhead



\multicolumn{5}{>{\raggedright}p{\dimexpr 6.27in+8\tabcolsep}}{\textcolor[HTML]{000000}{\textsuperscript{\fontsize{11}{11}\selectfont{1}}}\textcolor[HTML]{000000}{\fontsize{11}{11}\selectfont{CI\ =\ Confidence\ Interval}}} \\

\endfoot



\multicolumn{1}{>{\raggedright}p{\dimexpr 1.8in+0\tabcolsep}}{\textcolor[HTML]{000000}{\fontsize{11}{11}\selectfont{Sodium\ in\ mg}}} & \multicolumn{1}{>{\raggedright}p{\dimexpr 1.7in+0\tabcolsep}}{\textcolor[HTML]{000000}{\fontsize{11}{11}\selectfont{SurveyYear}}} & \multicolumn{1}{>{\centering}p{\dimexpr 0.68in+0\tabcolsep}}{\textcolor[HTML]{000000}{\fontsize{11}{11}\selectfont{-40}}} & \multicolumn{1}{>{\centering}p{\dimexpr 1.17in+0\tabcolsep}}{\textcolor[HTML]{000000}{\fontsize{11}{11}\selectfont{-50,\ -30}}} & \multicolumn{1}{>{\centering}p{\dimexpr 0.93in+0\tabcolsep}}{\textcolor[HTML]{000000}{\fontsize{11}{11}\selectfont{<0.001}}} \\





\multicolumn{1}{>{\raggedright}p{\dimexpr 1.8in+0\tabcolsep}}{\textcolor[HTML]{000000}{\fontsize{11}{11}\selectfont{Percent\ Energy\ UPF}}} & \multicolumn{1}{>{\raggedright}p{\dimexpr 1.7in+0\tabcolsep}}{\textcolor[HTML]{000000}{\fontsize{11}{11}\selectfont{SurveyYear}}} & \multicolumn{1}{>{\centering}p{\dimexpr 0.68in+0\tabcolsep}}{\textcolor[HTML]{000000}{\fontsize{11}{11}\selectfont{-0.57}}} & \multicolumn{1}{>{\centering}p{\dimexpr 1.17in+0\tabcolsep}}{\textcolor[HTML]{000000}{\fontsize{11}{11}\selectfont{-0.74,\ -0.39}}} & \multicolumn{1}{>{\centering}p{\dimexpr 0.93in+0\tabcolsep}}{\textcolor[HTML]{000000}{\fontsize{11}{11}\selectfont{<0.001}}} \\





\multicolumn{1}{>{\raggedright}p{\dimexpr 1.8in+0\tabcolsep}}{\textcolor[HTML]{000000}{\fontsize{11}{11}\selectfont{Systolic\ BP}}} & \multicolumn{1}{>{\raggedright}p{\dimexpr 1.7in+0\tabcolsep}}{\textcolor[HTML]{000000}{\fontsize{11}{11}\selectfont{NDNS\ Survey\ year}}} & \multicolumn{1}{>{\centering}p{\dimexpr 0.68in+0\tabcolsep}}{\textcolor[HTML]{000000}{\fontsize{11}{11}\selectfont{-0.37}}} & \multicolumn{1}{>{\centering}p{\dimexpr 1.17in+0\tabcolsep}}{\textcolor[HTML]{000000}{\fontsize{11}{11}\selectfont{-0.56,\ -0.19}}} & \multicolumn{1}{>{\centering}p{\dimexpr 0.93in+0\tabcolsep}}{\textcolor[HTML]{000000}{\fontsize{11}{11}\selectfont{<0.001}}} \\

\ascline{1pt}{000000}{1-5}



\end{longtable}



\arrayrulecolor[HTML]{000000}

\global\setlength{\arrayrulewidth}{\Oldarrayrulewidth}

\global\setlength{\tabcolsep}{\Oldtabcolsep}

\renewcommand*{\arraystretch}{1}

There is a change in mean systolic from 122-152 mmHg with a p value of
3.112e -7.

In summary there is statistically significant change in UPF and Na
intake and also in both systolic and diastolic pressures.

Has another factor affected the BP change ?

\hypertarget{comparatison-of-other-variables}{%
\subsubsection{Comparatison of other
variables}\label{comparatison-of-other-variables}}

How are variables distributed between the two cohorts. The NDNS dataset
was weighted to keep many of these the same between datasets. Continuous
variables are assessed using linear regression and categorical variables
using chi squared tests to give p.values.

Age and Sex The age of the two datasets does not show a statistically
significant change table x.

There is no statistically significant change in the sex distribution
over the years.

This might be due to differences in the numbers of excluded
participants. In particular there may be more younger people and women
taking e.g.~bblockers in one group.

This table suggests that there is a significant difference in the bmi of
the cohorts.

\global\setlength{\Oldarrayrulewidth}{\arrayrulewidth}

\global\setlength{\Oldtabcolsep}{\tabcolsep}

\setlength{\tabcolsep}{0pt}

\renewcommand*{\arraystretch}{1.5}



\providecommand{\ascline}[3]{\noalign{\global\arrayrulewidth #1}\arrayrulecolor[HTML]{#2}\cline{#3}}

\begin{longtable}[c]{|p{0.81in}|p{1.70in}|p{0.68in}|p{1.17in}|p{0.93in}}



\ascline{1pt}{000000}{1-5}

\multicolumn{1}{>{\raggedright}p{\dimexpr 0.81in+0\tabcolsep}}{\textcolor[HTML]{000000}{\fontsize{11}{11}\selectfont{\textbf{Group}}}} & \multicolumn{1}{>{\raggedright}p{\dimexpr 1.7in+0\tabcolsep}}{\textcolor[HTML]{000000}{\fontsize{11}{11}\selectfont{\textbf{Characteristic}}}} & \multicolumn{1}{>{\centering}p{\dimexpr 0.68in+0\tabcolsep}}{\textcolor[HTML]{000000}{\fontsize{11}{11}\selectfont{\textbf{Beta}}}} & \multicolumn{1}{>{\centering}p{\dimexpr 1.17in+0\tabcolsep}}{\textcolor[HTML]{000000}{\fontsize{11}{11}\selectfont{\textbf{95\%\ CI}}}\textcolor[HTML]{000000}{\textsuperscript{\fontsize{11}{11}\selectfont{1}}}} & \multicolumn{1}{>{\centering}p{\dimexpr 0.93in+0\tabcolsep}}{\textcolor[HTML]{000000}{\fontsize{11}{11}\selectfont{\textbf{p-value}}}} \\

\ascline{1pt}{000000}{1-5}\endhead



\multicolumn{5}{>{\raggedright}p{\dimexpr 5.29in+8\tabcolsep}}{\textcolor[HTML]{000000}{\textsuperscript{\fontsize{11}{11}\selectfont{1}}}\textcolor[HTML]{000000}{\fontsize{11}{11}\selectfont{CI\ =\ Confidence\ Interval}}} \\

\endfoot



\multicolumn{1}{>{\raggedright}p{\dimexpr 0.81in+0\tabcolsep}}{\textcolor[HTML]{000000}{\fontsize{11}{11}\selectfont{Age}}} & \multicolumn{1}{>{\raggedright}p{\dimexpr 1.7in+0\tabcolsep}}{\textcolor[HTML]{000000}{\fontsize{11}{11}\selectfont{NDNS\ Survey\ year}}} & \multicolumn{1}{>{\centering}p{\dimexpr 0.68in+0\tabcolsep}}{\textcolor[HTML]{000000}{\fontsize{11}{11}\selectfont{0.10}}} & \multicolumn{1}{>{\centering}p{\dimexpr 1.17in+0\tabcolsep}}{\textcolor[HTML]{000000}{\fontsize{11}{11}\selectfont{-0.06,\ 0.25}}} & \multicolumn{1}{>{\centering}p{\dimexpr 0.93in+0\tabcolsep}}{\textcolor[HTML]{000000}{\fontsize{11}{11}\selectfont{0.2}}} \\





\multicolumn{1}{>{\raggedright}p{\dimexpr 0.81in+0\tabcolsep}}{\textcolor[HTML]{000000}{\fontsize{11}{11}\selectfont{BMI}}} & \multicolumn{1}{>{\raggedright}p{\dimexpr 1.7in+0\tabcolsep}}{\textcolor[HTML]{000000}{\fontsize{11}{11}\selectfont{NDNS\ Survey\ year}}} & \multicolumn{1}{>{\centering}p{\dimexpr 0.68in+0\tabcolsep}}{\textcolor[HTML]{000000}{\fontsize{11}{11}\selectfont{-0.09}}} & \multicolumn{1}{>{\centering}p{\dimexpr 1.17in+0\tabcolsep}}{\textcolor[HTML]{000000}{\fontsize{11}{11}\selectfont{-0.13,\ -0.04}}} & \multicolumn{1}{>{\centering}p{\dimexpr 0.93in+0\tabcolsep}}{\textcolor[HTML]{000000}{\fontsize{11}{11}\selectfont{<0.001}}} \\

\ascline{1pt}{000000}{1-5}



\end{longtable}



\arrayrulecolor[HTML]{000000}

\global\setlength{\arrayrulewidth}{\Oldarrayrulewidth}

\global\setlength{\tabcolsep}{\Oldtabcolsep}

\renewcommand*{\arraystretch}{1}

There is a difference in the age of finishing education.

The differences in qimd, are not statistically significant.

These values identify a significant difference in the number of
vegetarians

\global\setlength{\Oldarrayrulewidth}{\arrayrulewidth}

\global\setlength{\Oldtabcolsep}{\tabcolsep}

\setlength{\tabcolsep}{0pt}

\renewcommand*{\arraystretch}{1.5}



\providecommand{\ascline}[3]{\noalign{\global\arrayrulewidth #1}\arrayrulecolor[HTML]{#2}\cline{#3}}

\begin{longtable}[c]{|p{1.12in}|p{0.90in}}

\caption{\textcolor[HTML]{000000}{\fontsize{11}{13}\selectfont{Categorical\ variables\ against\ Survey\ Year}}}\\

\ascline{2pt}{666666}{1-2}

\multicolumn{1}{>{\raggedright}p{\dimexpr 1.12in+0\tabcolsep}}{\textcolor[HTML]{000000}{\fontsize{11}{11}\selectfont{Variable}}} & \multicolumn{1}{>{\raggedleft}p{\dimexpr 0.9in+0\tabcolsep}}{\textcolor[HTML]{000000}{\fontsize{11}{11}\selectfont{p.value}}\textcolor[HTML]{000000}{\textsuperscript{\fontsize{11}{11}\selectfont{1}}}} \\

\ascline{2pt}{666666}{1-2}\endhead



\multicolumn{2}{>{\raggedright}p{\dimexpr 2.02in+2\tabcolsep}}{\textcolor[HTML]{000000}{\textsuperscript{\fontsize{11}{11}\selectfont{1}}}\textcolor[HTML]{000000}{\fontsize{11}{11}\selectfont{Chi\ Squared\ for\ categorical\ data}}} \\

\endfoot



\multicolumn{1}{>{\raggedright}p{\dimexpr 1.12in+0\tabcolsep}}{\textcolor[HTML]{000000}{\fontsize{11}{11}\selectfont{Sex}}} & \multicolumn{1}{>{\raggedleft}p{\dimexpr 0.9in+0\tabcolsep}}{\textcolor[HTML]{000000}{\fontsize{11}{11}\selectfont{0.5921}}} \\





\multicolumn{1}{>{\raggedright}p{\dimexpr 1.12in+0\tabcolsep}}{\textcolor[HTML]{000000}{\fontsize{11}{11}\selectfont{Education}}} & \multicolumn{1}{>{\raggedleft}p{\dimexpr 0.9in+0\tabcolsep}}{\textcolor[HTML]{000000}{\fontsize{11}{11}\selectfont{0.0000}}} \\





\multicolumn{1}{>{\raggedright}p{\dimexpr 1.12in+0\tabcolsep}}{\textcolor[HTML]{000000}{\fontsize{11}{11}\selectfont{IMD}}} & \multicolumn{1}{>{\raggedleft}p{\dimexpr 0.9in+0\tabcolsep}}{\textcolor[HTML]{000000}{\fontsize{11}{11}\selectfont{0.2208}}} \\





\multicolumn{1}{>{\raggedright}p{\dimexpr 1.12in+0\tabcolsep}}{\textcolor[HTML]{000000}{\fontsize{11}{11}\selectfont{Vegetarian}}} & \multicolumn{1}{>{\raggedleft}p{\dimexpr 0.9in+0\tabcolsep}}{\textcolor[HTML]{000000}{\fontsize{11}{11}\selectfont{0.0245}}} \\

\ascline{2pt}{666666}{1-2}



\end{longtable}



\arrayrulecolor[HTML]{000000}

\global\setlength{\arrayrulewidth}{\Oldarrayrulewidth}

\global\setlength{\tabcolsep}{\Oldtabcolsep}

\renewcommand*{\arraystretch}{1}

\hypertarget{regression-of-key-variables-on-systolic-bp}{%
\subsection{Regression of key variables on Systolic
BP}\label{regression-of-key-variables-on-systolic-bp}}

Simple linear regression equations look for the relationship between the
dependant variable, and the independent variable. For these I am looking
at the whole dataset

The regression models are examined for Sodium and UP against BP. These
use the populations where participants have been excluded. (analysis
including these makes no difference!!)

First, omsysval is compared to EnergykJ, then sodiummg.

\global\setlength{\Oldarrayrulewidth}{\arrayrulewidth}

\global\setlength{\Oldtabcolsep}{\tabcolsep}

\setlength{\tabcolsep}{0pt}

\renewcommand*{\arraystretch}{1.5}



\providecommand{\ascline}[3]{\noalign{\global\arrayrulewidth #1}\arrayrulecolor[HTML]{#2}\cline{#3}}

\begin{longtable}[c]{|p{0.92in}|p{1.99in}|p{0.68in}|p{1.17in}|p{0.93in}}

\caption{\textcolor[HTML]{000000}{\fontsize{11}{13}\selectfont{Univariable\ Regressions}}}\\

\ascline{1pt}{000000}{1-5}

\multicolumn{1}{>{\raggedright}p{\dimexpr 0.92in+0\tabcolsep}}{\textcolor[HTML]{000000}{\fontsize{11}{11}\selectfont{\textbf{Group}}}} & \multicolumn{1}{>{\raggedright}p{\dimexpr 1.99in+0\tabcolsep}}{\textcolor[HTML]{000000}{\fontsize{11}{11}\selectfont{\textbf{Characteristic}}}} & \multicolumn{1}{>{\centering}p{\dimexpr 0.68in+0\tabcolsep}}{\textcolor[HTML]{000000}{\fontsize{11}{11}\selectfont{\textbf{Beta}}}} & \multicolumn{1}{>{\centering}p{\dimexpr 1.17in+0\tabcolsep}}{\textcolor[HTML]{000000}{\fontsize{11}{11}\selectfont{\textbf{95\%\ CI}}}\textcolor[HTML]{000000}{\textsuperscript{\fontsize{11}{11}\selectfont{1}}}} & \multicolumn{1}{>{\centering}p{\dimexpr 0.93in+0\tabcolsep}}{\textcolor[HTML]{000000}{\fontsize{11}{11}\selectfont{\textbf{p-value}}}} \\

\ascline{1pt}{000000}{1-5}\endhead



\multicolumn{5}{>{\raggedright}p{\dimexpr 5.69in+8\tabcolsep}}{\textcolor[HTML]{000000}{\textsuperscript{\fontsize{11}{11}\selectfont{1}}}\textcolor[HTML]{000000}{\fontsize{11}{11}\selectfont{CI\ =\ Confidence\ Interval}}} \\

\endfoot



\multicolumn{1}{>{\raggedright}p{\dimexpr 0.92in+0\tabcolsep}}{\textcolor[HTML]{000000}{\fontsize{11}{11}\selectfont{Na/BP}}} & \multicolumn{1}{>{\raggedright}p{\dimexpr 1.99in+0\tabcolsep}}{\textcolor[HTML]{000000}{\fontsize{11}{11}\selectfont{Sodium\ (mg)\ diet\ only}}} & \multicolumn{1}{>{\centering}p{\dimexpr 0.68in+0\tabcolsep}}{\textcolor[HTML]{000000}{\fontsize{11}{11}\selectfont{0.00}}} & \multicolumn{1}{>{\centering}p{\dimexpr 1.17in+0\tabcolsep}}{\textcolor[HTML]{000000}{\fontsize{11}{11}\selectfont{0.00,\ 0.00}}} & \multicolumn{1}{>{\centering}p{\dimexpr 0.93in+0\tabcolsep}}{\textcolor[HTML]{000000}{\fontsize{11}{11}\selectfont{<0.001}}} \\





\multicolumn{1}{>{\raggedright}p{\dimexpr 0.92in+0\tabcolsep}}{\textcolor[HTML]{000000}{\fontsize{11}{11}\selectfont{UPF/bp}}} & \multicolumn{1}{>{\raggedright}p{\dimexpr 1.99in+0\tabcolsep}}{\textcolor[HTML]{000000}{\fontsize{11}{11}\selectfont{Epcnt\_4}}} & \multicolumn{1}{>{\centering}p{\dimexpr 0.68in+0\tabcolsep}}{\textcolor[HTML]{000000}{\fontsize{11}{11}\selectfont{-0.21}}} & \multicolumn{1}{>{\centering}p{\dimexpr 1.17in+0\tabcolsep}}{\textcolor[HTML]{000000}{\fontsize{11}{11}\selectfont{-0.24,\ -0.17}}} & \multicolumn{1}{>{\centering}p{\dimexpr 0.93in+0\tabcolsep}}{\textcolor[HTML]{000000}{\fontsize{11}{11}\selectfont{<0.001}}} \\





\multicolumn{1}{>{\raggedright}p{\dimexpr 0.92in+0\tabcolsep}}{\textcolor[HTML]{000000}{\fontsize{11}{11}\selectfont{UPF/Na}}} & \multicolumn{1}{>{\raggedright}p{\dimexpr 1.99in+0\tabcolsep}}{\textcolor[HTML]{000000}{\fontsize{11}{11}\selectfont{Sodium\ (mg)\ diet\ only}}} & \multicolumn{1}{>{\centering}p{\dimexpr 0.68in+0\tabcolsep}}{\textcolor[HTML]{000000}{\fontsize{11}{11}\selectfont{0.00}}} & \multicolumn{1}{>{\centering}p{\dimexpr 1.17in+0\tabcolsep}}{\textcolor[HTML]{000000}{\fontsize{11}{11}\selectfont{0.00,\ 0.00}}} & \multicolumn{1}{>{\centering}p{\dimexpr 0.93in+0\tabcolsep}}{\textcolor[HTML]{000000}{\fontsize{11}{11}\selectfont{<0.001}}} \\





\multicolumn{1}{>{\raggedright}p{\dimexpr 0.92in+0\tabcolsep}}{\textcolor[HTML]{000000}{\fontsize{11}{11}\selectfont{Age/BP}}} & \multicolumn{1}{>{\raggedright}p{\dimexpr 1.99in+0\tabcolsep}}{\textcolor[HTML]{000000}{\fontsize{11}{11}\selectfont{Age}}} & \multicolumn{1}{>{\centering}p{\dimexpr 0.68in+0\tabcolsep}}{\textcolor[HTML]{000000}{\fontsize{11}{11}\selectfont{0.43}}} & \multicolumn{1}{>{\centering}p{\dimexpr 1.17in+0\tabcolsep}}{\textcolor[HTML]{000000}{\fontsize{11}{11}\selectfont{0.40,\ 0.45}}} & \multicolumn{1}{>{\centering}p{\dimexpr 0.93in+0\tabcolsep}}{\textcolor[HTML]{000000}{\fontsize{11}{11}\selectfont{<0.001}}} \\





\multicolumn{1}{>{\raggedright}p{\dimexpr 0.92in+0\tabcolsep}}{\textcolor[HTML]{000000}{\fontsize{11}{11}\selectfont{Age/UPF}}} & \multicolumn{1}{>{\raggedright}p{\dimexpr 1.99in+0\tabcolsep}}{\textcolor[HTML]{000000}{\fontsize{11}{11}\selectfont{Age}}} & \multicolumn{1}{>{\centering}p{\dimexpr 0.68in+0\tabcolsep}}{\textcolor[HTML]{000000}{\fontsize{11}{11}\selectfont{-0.21}}} & \multicolumn{1}{>{\centering}p{\dimexpr 1.17in+0\tabcolsep}}{\textcolor[HTML]{000000}{\fontsize{11}{11}\selectfont{-0.23,\ -0.20}}} & \multicolumn{1}{>{\centering}p{\dimexpr 0.93in+0\tabcolsep}}{\textcolor[HTML]{000000}{\fontsize{11}{11}\selectfont{<0.001}}} \\





\multicolumn{1}{>{\raggedright}p{\dimexpr 0.92in+0\tabcolsep}}{\textcolor[HTML]{000000}{\fontsize{11}{11}\selectfont{Age/Na}}} & \multicolumn{1}{>{\raggedright}p{\dimexpr 1.99in+0\tabcolsep}}{\textcolor[HTML]{000000}{\fontsize{11}{11}\selectfont{Age}}} & \multicolumn{1}{>{\centering}p{\dimexpr 0.68in+0\tabcolsep}}{\textcolor[HTML]{000000}{\fontsize{11}{11}\selectfont{0.75}}} & \multicolumn{1}{>{\centering}p{\dimexpr 1.17in+0\tabcolsep}}{\textcolor[HTML]{000000}{\fontsize{11}{11}\selectfont{0.08,\ 1.4}}} & \multicolumn{1}{>{\centering}p{\dimexpr 0.93in+0\tabcolsep}}{\textcolor[HTML]{000000}{\fontsize{11}{11}\selectfont{0.028}}} \\

\ascline{1pt}{000000}{1-5}



\end{longtable}



\arrayrulecolor[HTML]{000000}

\global\setlength{\arrayrulewidth}{\Oldarrayrulewidth}

\global\setlength{\tabcolsep}{\Oldtabcolsep}

\renewcommand*{\arraystretch}{1}

Sodium intake appears to have no linear relationship with BP or UPF
energy intake. The UPF energy intake has a negative relationship with
BP. BP clearly increases with age. Regression of UPF with age shows an
equal and opposite affect to that on BP. Regression of Sodium with age
shows a weaker positive effect.

In conclusion the linear regression models show that there are
correlations between the systolic BP and energy intake only. The next
section will examine how this situation changes as variables interact in
more complex models.

\hypertarget{multi-variable-regression-on-systolic-bp}{%
\subsection{Multi variable regression on Systolic
BP}\label{multi-variable-regression-on-systolic-bp}}

This uses a model of several variables and it can highlight the
contributions of each variable. The intention is to develop an optimal
model which mathematically describes the situation.

In particular the research question asks about the relationship between
Sodium and UPF intake with BP. The models will reflect this question
with models looking to include or exclude particular variables.
Comparisons between these models are then made using sensitivity
analysis, identifying how sensitive the model is to sodium, or other
factors

This first model looks at the relationships between BP and Age and Sex
education and IMD all of which may have an effect on BP. This model
excludes UPF and Na.

This first model shows that all these variables, Age, Sex, education,
IMD, and bmi, give statistically significant coefficients for the model
which suggests that they do have an important part to play in any
optimal model.

The next model adds Sodiummg.

This second model gives Sodiummg, educfinh, and IMD statistical
significance. VitaminD shows no statistical significance, TotalEMJ and
sqrt(pcnt) and ethgrp2 all have limited significance.

Now we add UPF as total energy from nova 4 or UPF

UPF does not seem significant\ldots{}

but when removing sodiummg

the UPF becomes significant! This suggests that the effect of UPF is
mediated by Sodium!!

comparing AIC for these three models

\global\setlength{\Oldarrayrulewidth}{\arrayrulewidth}

\global\setlength{\Oldtabcolsep}{\tabcolsep}

\setlength{\tabcolsep}{0pt}

\renewcommand*{\arraystretch}{1.5}



\providecommand{\ascline}[3]{\noalign{\global\arrayrulewidth #1}\arrayrulecolor[HTML]{#2}\cline{#3}}

\begin{longtable}[c]{|p{1.99in}|p{0.68in}|p{1.08in}|p{0.93in}|p{0.68in}|p{1.08in}|p{0.93in}|p{0.68in}|p{1.12in}|p{0.93in}|p{0.68in}|p{1.12in}|p{0.93in}}



\ascline{1pt}{000000}{1-13}

\multicolumn{1}{>{\raggedright}p{\dimexpr 1.99in+0\tabcolsep}}{\textcolor[HTML]{000000}{\fontsize{11}{11}\selectfont{\ }}} & \multicolumn{3}{>{\centering}p{\dimexpr 2.68in+4\tabcolsep}}{\textcolor[HTML]{000000}{\fontsize{11}{11}\selectfont{No\ sodium\ or\ UPF}}} & \multicolumn{3}{>{\centering}p{\dimexpr 2.68in+4\tabcolsep}}{\textcolor[HTML]{000000}{\fontsize{11}{11}\selectfont{Sodium\ only}}} & \multicolumn{3}{>{\centering}p{\dimexpr 2.72in+4\tabcolsep}}{\textcolor[HTML]{000000}{\fontsize{11}{11}\selectfont{Epcnt\ only}}} & \multicolumn{3}{>{\centering}p{\dimexpr 2.72in+4\tabcolsep}}{\textcolor[HTML]{000000}{\fontsize{11}{11}\selectfont{Sodium\ and\ UPF}}} \\

\ascline{1pt}{000000}{1-13}



\multicolumn{1}{>{\raggedright}p{\dimexpr 1.99in+0\tabcolsep}}{\textcolor[HTML]{000000}{\fontsize{11}{11}\selectfont{\textbf{Characteristic}}}} & \multicolumn{1}{>{\centering}p{\dimexpr 0.68in+0\tabcolsep}}{\textcolor[HTML]{000000}{\fontsize{11}{11}\selectfont{\textbf{Beta}}}} & \multicolumn{1}{>{\centering}p{\dimexpr 1.08in+0\tabcolsep}}{\textcolor[HTML]{000000}{\fontsize{11}{11}\selectfont{\textbf{95\%\ CI}}}\textcolor[HTML]{000000}{\textsuperscript{\fontsize{11}{11}\selectfont{1}}}} & \multicolumn{1}{>{\centering}p{\dimexpr 0.93in+0\tabcolsep}}{\textcolor[HTML]{000000}{\fontsize{11}{11}\selectfont{\textbf{p-value}}}} & \multicolumn{1}{>{\centering}p{\dimexpr 0.68in+0\tabcolsep}}{\textcolor[HTML]{000000}{\fontsize{11}{11}\selectfont{\textbf{Beta}}}} & \multicolumn{1}{>{\centering}p{\dimexpr 1.08in+0\tabcolsep}}{\textcolor[HTML]{000000}{\fontsize{11}{11}\selectfont{\textbf{95\%\ CI}}}\textcolor[HTML]{000000}{\textsuperscript{\fontsize{11}{11}\selectfont{1}}}} & \multicolumn{1}{>{\centering}p{\dimexpr 0.93in+0\tabcolsep}}{\textcolor[HTML]{000000}{\fontsize{11}{11}\selectfont{\textbf{p-value}}}} & \multicolumn{1}{>{\centering}p{\dimexpr 0.68in+0\tabcolsep}}{\textcolor[HTML]{000000}{\fontsize{11}{11}\selectfont{\textbf{Beta}}}} & \multicolumn{1}{>{\centering}p{\dimexpr 1.12in+0\tabcolsep}}{\textcolor[HTML]{000000}{\fontsize{11}{11}\selectfont{\textbf{95\%\ CI}}}\textcolor[HTML]{000000}{\textsuperscript{\fontsize{11}{11}\selectfont{1}}}} & \multicolumn{1}{>{\centering}p{\dimexpr 0.93in+0\tabcolsep}}{\textcolor[HTML]{000000}{\fontsize{11}{11}\selectfont{\textbf{p-value}}}} & \multicolumn{1}{>{\centering}p{\dimexpr 0.68in+0\tabcolsep}}{\textcolor[HTML]{000000}{\fontsize{11}{11}\selectfont{\textbf{Beta}}}} & \multicolumn{1}{>{\centering}p{\dimexpr 1.12in+0\tabcolsep}}{\textcolor[HTML]{000000}{\fontsize{11}{11}\selectfont{\textbf{95\%\ CI}}}\textcolor[HTML]{000000}{\textsuperscript{\fontsize{11}{11}\selectfont{1}}}} & \multicolumn{1}{>{\centering}p{\dimexpr 0.93in+0\tabcolsep}}{\textcolor[HTML]{000000}{\fontsize{11}{11}\selectfont{\textbf{p-value}}}} \\

\ascline{1pt}{000000}{1-13}\endhead



\multicolumn{13}{>{\raggedright}p{\dimexpr 12.8in+24\tabcolsep}}{\textcolor[HTML]{000000}{\textsuperscript{\fontsize{11}{11}\selectfont{1}}}\textcolor[HTML]{000000}{\fontsize{11}{11}\selectfont{CI\ =\ Confidence\ Interval}}} \\

\endfoot



\multicolumn{1}{>{\raggedright}p{\dimexpr 1.99in+0\tabcolsep}}{\textcolor[HTML]{000000}{\fontsize{11}{11}\selectfont{Age}}} & \multicolumn{1}{>{\centering}p{\dimexpr 0.68in+0\tabcolsep}}{\textcolor[HTML]{000000}{\fontsize{11}{11}\selectfont{0.38}}} & \multicolumn{1}{>{\centering}p{\dimexpr 1.08in+0\tabcolsep}}{\textcolor[HTML]{000000}{\fontsize{11}{11}\selectfont{0.34,\ 0.42}}} & \multicolumn{1}{>{\centering}p{\dimexpr 0.93in+0\tabcolsep}}{\textcolor[HTML]{000000}{\fontsize{11}{11}\selectfont{<0.001}}} & \multicolumn{1}{>{\centering}p{\dimexpr 0.68in+0\tabcolsep}}{\textcolor[HTML]{000000}{\fontsize{11}{11}\selectfont{0.38}}} & \multicolumn{1}{>{\centering}p{\dimexpr 1.08in+0\tabcolsep}}{\textcolor[HTML]{000000}{\fontsize{11}{11}\selectfont{0.34,\ 0.42}}} & \multicolumn{1}{>{\centering}p{\dimexpr 0.93in+0\tabcolsep}}{\textcolor[HTML]{000000}{\fontsize{11}{11}\selectfont{<0.001}}} & \multicolumn{1}{>{\centering}p{\dimexpr 0.68in+0\tabcolsep}}{\textcolor[HTML]{000000}{\fontsize{11}{11}\selectfont{0.37}}} & \multicolumn{1}{>{\centering}p{\dimexpr 1.12in+0\tabcolsep}}{\textcolor[HTML]{000000}{\fontsize{11}{11}\selectfont{0.33,\ 0.42}}} & \multicolumn{1}{>{\centering}p{\dimexpr 0.93in+0\tabcolsep}}{\textcolor[HTML]{000000}{\fontsize{11}{11}\selectfont{<0.001}}} & \multicolumn{1}{>{\centering}p{\dimexpr 0.68in+0\tabcolsep}}{\textcolor[HTML]{000000}{\fontsize{11}{11}\selectfont{0.37}}} & \multicolumn{1}{>{\centering}p{\dimexpr 1.12in+0\tabcolsep}}{\textcolor[HTML]{000000}{\fontsize{11}{11}\selectfont{0.33,\ 0.42}}} & \multicolumn{1}{>{\centering}p{\dimexpr 0.93in+0\tabcolsep}}{\textcolor[HTML]{000000}{\fontsize{11}{11}\selectfont{<0.001}}} \\





\multicolumn{1}{>{\raggedright}p{\dimexpr 1.99in+0\tabcolsep}}{\textcolor[HTML]{000000}{\fontsize{11}{11}\selectfont{Sex}}} & \multicolumn{1}{>{\centering}p{\dimexpr 0.68in+0\tabcolsep}}{\textcolor[HTML]{000000}{\fontsize{11}{11}\selectfont{}}} & \multicolumn{1}{>{\centering}p{\dimexpr 1.08in+0\tabcolsep}}{\textcolor[HTML]{000000}{\fontsize{11}{11}\selectfont{}}} & \multicolumn{1}{>{\centering}p{\dimexpr 0.93in+0\tabcolsep}}{\textcolor[HTML]{000000}{\fontsize{11}{11}\selectfont{}}} & \multicolumn{1}{>{\centering}p{\dimexpr 0.68in+0\tabcolsep}}{\textcolor[HTML]{000000}{\fontsize{11}{11}\selectfont{}}} & \multicolumn{1}{>{\centering}p{\dimexpr 1.08in+0\tabcolsep}}{\textcolor[HTML]{000000}{\fontsize{11}{11}\selectfont{}}} & \multicolumn{1}{>{\centering}p{\dimexpr 0.93in+0\tabcolsep}}{\textcolor[HTML]{000000}{\fontsize{11}{11}\selectfont{}}} & \multicolumn{1}{>{\centering}p{\dimexpr 0.68in+0\tabcolsep}}{\textcolor[HTML]{000000}{\fontsize{11}{11}\selectfont{}}} & \multicolumn{1}{>{\centering}p{\dimexpr 1.12in+0\tabcolsep}}{\textcolor[HTML]{000000}{\fontsize{11}{11}\selectfont{}}} & \multicolumn{1}{>{\centering}p{\dimexpr 0.93in+0\tabcolsep}}{\textcolor[HTML]{000000}{\fontsize{11}{11}\selectfont{}}} & \multicolumn{1}{>{\centering}p{\dimexpr 0.68in+0\tabcolsep}}{\textcolor[HTML]{000000}{\fontsize{11}{11}\selectfont{}}} & \multicolumn{1}{>{\centering}p{\dimexpr 1.12in+0\tabcolsep}}{\textcolor[HTML]{000000}{\fontsize{11}{11}\selectfont{}}} & \multicolumn{1}{>{\centering}p{\dimexpr 0.93in+0\tabcolsep}}{\textcolor[HTML]{000000}{\fontsize{11}{11}\selectfont{}}} \\





\multicolumn{1}{>{\raggedright}p{\dimexpr 1.99in+0\tabcolsep}}{\textcolor[HTML]{000000}{\fontsize{11}{11}\selectfont{Male}}} & \multicolumn{1}{>{\centering}p{\dimexpr 0.68in+0\tabcolsep}}{\textcolor[HTML]{000000}{\fontsize{11}{11}\selectfont{—}}} & \multicolumn{1}{>{\centering}p{\dimexpr 1.08in+0\tabcolsep}}{\textcolor[HTML]{000000}{\fontsize{11}{11}\selectfont{—}}} & \multicolumn{1}{>{\centering}p{\dimexpr 0.93in+0\tabcolsep}}{\textcolor[HTML]{000000}{\fontsize{11}{11}\selectfont{}}} & \multicolumn{1}{>{\centering}p{\dimexpr 0.68in+0\tabcolsep}}{\textcolor[HTML]{000000}{\fontsize{11}{11}\selectfont{—}}} & \multicolumn{1}{>{\centering}p{\dimexpr 1.08in+0\tabcolsep}}{\textcolor[HTML]{000000}{\fontsize{11}{11}\selectfont{—}}} & \multicolumn{1}{>{\centering}p{\dimexpr 0.93in+0\tabcolsep}}{\textcolor[HTML]{000000}{\fontsize{11}{11}\selectfont{}}} & \multicolumn{1}{>{\centering}p{\dimexpr 0.68in+0\tabcolsep}}{\textcolor[HTML]{000000}{\fontsize{11}{11}\selectfont{—}}} & \multicolumn{1}{>{\centering}p{\dimexpr 1.12in+0\tabcolsep}}{\textcolor[HTML]{000000}{\fontsize{11}{11}\selectfont{—}}} & \multicolumn{1}{>{\centering}p{\dimexpr 0.93in+0\tabcolsep}}{\textcolor[HTML]{000000}{\fontsize{11}{11}\selectfont{}}} & \multicolumn{1}{>{\centering}p{\dimexpr 0.68in+0\tabcolsep}}{\textcolor[HTML]{000000}{\fontsize{11}{11}\selectfont{—}}} & \multicolumn{1}{>{\centering}p{\dimexpr 1.12in+0\tabcolsep}}{\textcolor[HTML]{000000}{\fontsize{11}{11}\selectfont{—}}} & \multicolumn{1}{>{\centering}p{\dimexpr 0.93in+0\tabcolsep}}{\textcolor[HTML]{000000}{\fontsize{11}{11}\selectfont{}}} \\





\multicolumn{1}{>{\raggedright}p{\dimexpr 1.99in+0\tabcolsep}}{\textcolor[HTML]{000000}{\fontsize{11}{11}\selectfont{Female}}} & \multicolumn{1}{>{\centering}p{\dimexpr 0.68in+0\tabcolsep}}{\textcolor[HTML]{000000}{\fontsize{11}{11}\selectfont{-6.3}}} & \multicolumn{1}{>{\centering}p{\dimexpr 1.08in+0\tabcolsep}}{\textcolor[HTML]{000000}{\fontsize{11}{11}\selectfont{-7.5,\ -5.1}}} & \multicolumn{1}{>{\centering}p{\dimexpr 0.93in+0\tabcolsep}}{\textcolor[HTML]{000000}{\fontsize{11}{11}\selectfont{<0.001}}} & \multicolumn{1}{>{\centering}p{\dimexpr 0.68in+0\tabcolsep}}{\textcolor[HTML]{000000}{\fontsize{11}{11}\selectfont{-5.8}}} & \multicolumn{1}{>{\centering}p{\dimexpr 1.08in+0\tabcolsep}}{\textcolor[HTML]{000000}{\fontsize{11}{11}\selectfont{-7.1,\ -4.5}}} & \multicolumn{1}{>{\centering}p{\dimexpr 0.93in+0\tabcolsep}}{\textcolor[HTML]{000000}{\fontsize{11}{11}\selectfont{<0.001}}} & \multicolumn{1}{>{\centering}p{\dimexpr 0.68in+0\tabcolsep}}{\textcolor[HTML]{000000}{\fontsize{11}{11}\selectfont{-6.3}}} & \multicolumn{1}{>{\centering}p{\dimexpr 1.12in+0\tabcolsep}}{\textcolor[HTML]{000000}{\fontsize{11}{11}\selectfont{-7.5,\ -5.1}}} & \multicolumn{1}{>{\centering}p{\dimexpr 0.93in+0\tabcolsep}}{\textcolor[HTML]{000000}{\fontsize{11}{11}\selectfont{<0.001}}} & \multicolumn{1}{>{\centering}p{\dimexpr 0.68in+0\tabcolsep}}{\textcolor[HTML]{000000}{\fontsize{11}{11}\selectfont{-5.8}}} & \multicolumn{1}{>{\centering}p{\dimexpr 1.12in+0\tabcolsep}}{\textcolor[HTML]{000000}{\fontsize{11}{11}\selectfont{-7.1,\ -4.5}}} & \multicolumn{1}{>{\centering}p{\dimexpr 0.93in+0\tabcolsep}}{\textcolor[HTML]{000000}{\fontsize{11}{11}\selectfont{<0.001}}} \\





\multicolumn{1}{>{\raggedright}p{\dimexpr 1.99in+0\tabcolsep}}{\textcolor[HTML]{000000}{\fontsize{11}{11}\selectfont{(D)\ Valid\ BMI}}} & \multicolumn{1}{>{\centering}p{\dimexpr 0.68in+0\tabcolsep}}{\textcolor[HTML]{000000}{\fontsize{11}{11}\selectfont{0.40}}} & \multicolumn{1}{>{\centering}p{\dimexpr 1.08in+0\tabcolsep}}{\textcolor[HTML]{000000}{\fontsize{11}{11}\selectfont{0.24,\ 0.56}}} & \multicolumn{1}{>{\centering}p{\dimexpr 0.93in+0\tabcolsep}}{\textcolor[HTML]{000000}{\fontsize{11}{11}\selectfont{<0.001}}} & \multicolumn{1}{>{\centering}p{\dimexpr 0.68in+0\tabcolsep}}{\textcolor[HTML]{000000}{\fontsize{11}{11}\selectfont{0.39}}} & \multicolumn{1}{>{\centering}p{\dimexpr 1.08in+0\tabcolsep}}{\textcolor[HTML]{000000}{\fontsize{11}{11}\selectfont{0.23,\ 0.55}}} & \multicolumn{1}{>{\centering}p{\dimexpr 0.93in+0\tabcolsep}}{\textcolor[HTML]{000000}{\fontsize{11}{11}\selectfont{<0.001}}} & \multicolumn{1}{>{\centering}p{\dimexpr 0.68in+0\tabcolsep}}{\textcolor[HTML]{000000}{\fontsize{11}{11}\selectfont{0.40}}} & \multicolumn{1}{>{\centering}p{\dimexpr 1.12in+0\tabcolsep}}{\textcolor[HTML]{000000}{\fontsize{11}{11}\selectfont{0.24,\ 0.56}}} & \multicolumn{1}{>{\centering}p{\dimexpr 0.93in+0\tabcolsep}}{\textcolor[HTML]{000000}{\fontsize{11}{11}\selectfont{<0.001}}} & \multicolumn{1}{>{\centering}p{\dimexpr 0.68in+0\tabcolsep}}{\textcolor[HTML]{000000}{\fontsize{11}{11}\selectfont{0.39}}} & \multicolumn{1}{>{\centering}p{\dimexpr 1.12in+0\tabcolsep}}{\textcolor[HTML]{000000}{\fontsize{11}{11}\selectfont{0.23,\ 0.55}}} & \multicolumn{1}{>{\centering}p{\dimexpr 0.93in+0\tabcolsep}}{\textcolor[HTML]{000000}{\fontsize{11}{11}\selectfont{<0.001}}} \\





\multicolumn{1}{>{\raggedright}p{\dimexpr 1.99in+0\tabcolsep}}{\textcolor[HTML]{000000}{\fontsize{11}{11}\selectfont{educfinh}}} & \multicolumn{1}{>{\centering}p{\dimexpr 0.68in+0\tabcolsep}}{\textcolor[HTML]{000000}{\fontsize{11}{11}\selectfont{}}} & \multicolumn{1}{>{\centering}p{\dimexpr 1.08in+0\tabcolsep}}{\textcolor[HTML]{000000}{\fontsize{11}{11}\selectfont{}}} & \multicolumn{1}{>{\centering}p{\dimexpr 0.93in+0\tabcolsep}}{\textcolor[HTML]{000000}{\fontsize{11}{11}\selectfont{}}} & \multicolumn{1}{>{\centering}p{\dimexpr 0.68in+0\tabcolsep}}{\textcolor[HTML]{000000}{\fontsize{11}{11}\selectfont{}}} & \multicolumn{1}{>{\centering}p{\dimexpr 1.08in+0\tabcolsep}}{\textcolor[HTML]{000000}{\fontsize{11}{11}\selectfont{}}} & \multicolumn{1}{>{\centering}p{\dimexpr 0.93in+0\tabcolsep}}{\textcolor[HTML]{000000}{\fontsize{11}{11}\selectfont{}}} & \multicolumn{1}{>{\centering}p{\dimexpr 0.68in+0\tabcolsep}}{\textcolor[HTML]{000000}{\fontsize{11}{11}\selectfont{}}} & \multicolumn{1}{>{\centering}p{\dimexpr 1.12in+0\tabcolsep}}{\textcolor[HTML]{000000}{\fontsize{11}{11}\selectfont{}}} & \multicolumn{1}{>{\centering}p{\dimexpr 0.93in+0\tabcolsep}}{\textcolor[HTML]{000000}{\fontsize{11}{11}\selectfont{}}} & \multicolumn{1}{>{\centering}p{\dimexpr 0.68in+0\tabcolsep}}{\textcolor[HTML]{000000}{\fontsize{11}{11}\selectfont{}}} & \multicolumn{1}{>{\centering}p{\dimexpr 1.12in+0\tabcolsep}}{\textcolor[HTML]{000000}{\fontsize{11}{11}\selectfont{}}} & \multicolumn{1}{>{\centering}p{\dimexpr 0.93in+0\tabcolsep}}{\textcolor[HTML]{000000}{\fontsize{11}{11}\selectfont{}}} \\





\multicolumn{1}{>{\raggedright}p{\dimexpr 1.99in+0\tabcolsep}}{\textcolor[HTML]{000000}{\fontsize{11}{11}\selectfont{1}}} & \multicolumn{1}{>{\centering}p{\dimexpr 0.68in+0\tabcolsep}}{\textcolor[HTML]{000000}{\fontsize{11}{11}\selectfont{—}}} & \multicolumn{1}{>{\centering}p{\dimexpr 1.08in+0\tabcolsep}}{\textcolor[HTML]{000000}{\fontsize{11}{11}\selectfont{—}}} & \multicolumn{1}{>{\centering}p{\dimexpr 0.93in+0\tabcolsep}}{\textcolor[HTML]{000000}{\fontsize{11}{11}\selectfont{}}} & \multicolumn{1}{>{\centering}p{\dimexpr 0.68in+0\tabcolsep}}{\textcolor[HTML]{000000}{\fontsize{11}{11}\selectfont{—}}} & \multicolumn{1}{>{\centering}p{\dimexpr 1.08in+0\tabcolsep}}{\textcolor[HTML]{000000}{\fontsize{11}{11}\selectfont{—}}} & \multicolumn{1}{>{\centering}p{\dimexpr 0.93in+0\tabcolsep}}{\textcolor[HTML]{000000}{\fontsize{11}{11}\selectfont{}}} & \multicolumn{1}{>{\centering}p{\dimexpr 0.68in+0\tabcolsep}}{\textcolor[HTML]{000000}{\fontsize{11}{11}\selectfont{—}}} & \multicolumn{1}{>{\centering}p{\dimexpr 1.12in+0\tabcolsep}}{\textcolor[HTML]{000000}{\fontsize{11}{11}\selectfont{—}}} & \multicolumn{1}{>{\centering}p{\dimexpr 0.93in+0\tabcolsep}}{\textcolor[HTML]{000000}{\fontsize{11}{11}\selectfont{}}} & \multicolumn{1}{>{\centering}p{\dimexpr 0.68in+0\tabcolsep}}{\textcolor[HTML]{000000}{\fontsize{11}{11}\selectfont{—}}} & \multicolumn{1}{>{\centering}p{\dimexpr 1.12in+0\tabcolsep}}{\textcolor[HTML]{000000}{\fontsize{11}{11}\selectfont{—}}} & \multicolumn{1}{>{\centering}p{\dimexpr 0.93in+0\tabcolsep}}{\textcolor[HTML]{000000}{\fontsize{11}{11}\selectfont{}}} \\





\multicolumn{1}{>{\raggedright}p{\dimexpr 1.99in+0\tabcolsep}}{\textcolor[HTML]{000000}{\fontsize{11}{11}\selectfont{2}}} & \multicolumn{1}{>{\centering}p{\dimexpr 0.68in+0\tabcolsep}}{\textcolor[HTML]{000000}{\fontsize{11}{11}\selectfont{1.4}}} & \multicolumn{1}{>{\centering}p{\dimexpr 1.08in+0\tabcolsep}}{\textcolor[HTML]{000000}{\fontsize{11}{11}\selectfont{-4.9,\ 7.7}}} & \multicolumn{1}{>{\centering}p{\dimexpr 0.93in+0\tabcolsep}}{\textcolor[HTML]{000000}{\fontsize{11}{11}\selectfont{0.7}}} & \multicolumn{1}{>{\centering}p{\dimexpr 0.68in+0\tabcolsep}}{\textcolor[HTML]{000000}{\fontsize{11}{11}\selectfont{2.0}}} & \multicolumn{1}{>{\centering}p{\dimexpr 1.08in+0\tabcolsep}}{\textcolor[HTML]{000000}{\fontsize{11}{11}\selectfont{-4.0,\ 8.0}}} & \multicolumn{1}{>{\centering}p{\dimexpr 0.93in+0\tabcolsep}}{\textcolor[HTML]{000000}{\fontsize{11}{11}\selectfont{0.5}}} & \multicolumn{1}{>{\centering}p{\dimexpr 0.68in+0\tabcolsep}}{\textcolor[HTML]{000000}{\fontsize{11}{11}\selectfont{1.3}}} & \multicolumn{1}{>{\centering}p{\dimexpr 1.12in+0\tabcolsep}}{\textcolor[HTML]{000000}{\fontsize{11}{11}\selectfont{-5.1,\ 7.6}}} & \multicolumn{1}{>{\centering}p{\dimexpr 0.93in+0\tabcolsep}}{\textcolor[HTML]{000000}{\fontsize{11}{11}\selectfont{0.7}}} & \multicolumn{1}{>{\centering}p{\dimexpr 0.68in+0\tabcolsep}}{\textcolor[HTML]{000000}{\fontsize{11}{11}\selectfont{1.7}}} & \multicolumn{1}{>{\centering}p{\dimexpr 1.12in+0\tabcolsep}}{\textcolor[HTML]{000000}{\fontsize{11}{11}\selectfont{-4.4,\ 7.8}}} & \multicolumn{1}{>{\centering}p{\dimexpr 0.93in+0\tabcolsep}}{\textcolor[HTML]{000000}{\fontsize{11}{11}\selectfont{0.6}}} \\





\multicolumn{1}{>{\raggedright}p{\dimexpr 1.99in+0\tabcolsep}}{\textcolor[HTML]{000000}{\fontsize{11}{11}\selectfont{3}}} & \multicolumn{1}{>{\centering}p{\dimexpr 0.68in+0\tabcolsep}}{\textcolor[HTML]{000000}{\fontsize{11}{11}\selectfont{0.68}}} & \multicolumn{1}{>{\centering}p{\dimexpr 1.08in+0\tabcolsep}}{\textcolor[HTML]{000000}{\fontsize{11}{11}\selectfont{-4.1,\ 5.4}}} & \multicolumn{1}{>{\centering}p{\dimexpr 0.93in+0\tabcolsep}}{\textcolor[HTML]{000000}{\fontsize{11}{11}\selectfont{0.8}}} & \multicolumn{1}{>{\centering}p{\dimexpr 0.68in+0\tabcolsep}}{\textcolor[HTML]{000000}{\fontsize{11}{11}\selectfont{1.0}}} & \multicolumn{1}{>{\centering}p{\dimexpr 1.08in+0\tabcolsep}}{\textcolor[HTML]{000000}{\fontsize{11}{11}\selectfont{-3.7,\ 5.7}}} & \multicolumn{1}{>{\centering}p{\dimexpr 0.93in+0\tabcolsep}}{\textcolor[HTML]{000000}{\fontsize{11}{11}\selectfont{0.7}}} & \multicolumn{1}{>{\centering}p{\dimexpr 0.68in+0\tabcolsep}}{\textcolor[HTML]{000000}{\fontsize{11}{11}\selectfont{0.69}}} & \multicolumn{1}{>{\centering}p{\dimexpr 1.12in+0\tabcolsep}}{\textcolor[HTML]{000000}{\fontsize{11}{11}\selectfont{-4.1,\ 5.4}}} & \multicolumn{1}{>{\centering}p{\dimexpr 0.93in+0\tabcolsep}}{\textcolor[HTML]{000000}{\fontsize{11}{11}\selectfont{0.8}}} & \multicolumn{1}{>{\centering}p{\dimexpr 0.68in+0\tabcolsep}}{\textcolor[HTML]{000000}{\fontsize{11}{11}\selectfont{1.0}}} & \multicolumn{1}{>{\centering}p{\dimexpr 1.12in+0\tabcolsep}}{\textcolor[HTML]{000000}{\fontsize{11}{11}\selectfont{-3.7,\ 5.8}}} & \multicolumn{1}{>{\centering}p{\dimexpr 0.93in+0\tabcolsep}}{\textcolor[HTML]{000000}{\fontsize{11}{11}\selectfont{0.7}}} \\





\multicolumn{1}{>{\raggedright}p{\dimexpr 1.99in+0\tabcolsep}}{\textcolor[HTML]{000000}{\fontsize{11}{11}\selectfont{4}}} & \multicolumn{1}{>{\centering}p{\dimexpr 0.68in+0\tabcolsep}}{\textcolor[HTML]{000000}{\fontsize{11}{11}\selectfont{-1.4}}} & \multicolumn{1}{>{\centering}p{\dimexpr 1.08in+0\tabcolsep}}{\textcolor[HTML]{000000}{\fontsize{11}{11}\selectfont{-4.6,\ 1.9}}} & \multicolumn{1}{>{\centering}p{\dimexpr 0.93in+0\tabcolsep}}{\textcolor[HTML]{000000}{\fontsize{11}{11}\selectfont{0.4}}} & \multicolumn{1}{>{\centering}p{\dimexpr 0.68in+0\tabcolsep}}{\textcolor[HTML]{000000}{\fontsize{11}{11}\selectfont{-1.2}}} & \multicolumn{1}{>{\centering}p{\dimexpr 1.08in+0\tabcolsep}}{\textcolor[HTML]{000000}{\fontsize{11}{11}\selectfont{-4.5,\ 2.0}}} & \multicolumn{1}{>{\centering}p{\dimexpr 0.93in+0\tabcolsep}}{\textcolor[HTML]{000000}{\fontsize{11}{11}\selectfont{0.5}}} & \multicolumn{1}{>{\centering}p{\dimexpr 0.68in+0\tabcolsep}}{\textcolor[HTML]{000000}{\fontsize{11}{11}\selectfont{-1.4}}} & \multicolumn{1}{>{\centering}p{\dimexpr 1.12in+0\tabcolsep}}{\textcolor[HTML]{000000}{\fontsize{11}{11}\selectfont{-4.7,\ 1.9}}} & \multicolumn{1}{>{\centering}p{\dimexpr 0.93in+0\tabcolsep}}{\textcolor[HTML]{000000}{\fontsize{11}{11}\selectfont{0.4}}} & \multicolumn{1}{>{\centering}p{\dimexpr 0.68in+0\tabcolsep}}{\textcolor[HTML]{000000}{\fontsize{11}{11}\selectfont{-1.2}}} & \multicolumn{1}{>{\centering}p{\dimexpr 1.12in+0\tabcolsep}}{\textcolor[HTML]{000000}{\fontsize{11}{11}\selectfont{-4.5,\ 2.0}}} & \multicolumn{1}{>{\centering}p{\dimexpr 0.93in+0\tabcolsep}}{\textcolor[HTML]{000000}{\fontsize{11}{11}\selectfont{0.5}}} \\





\multicolumn{1}{>{\raggedright}p{\dimexpr 1.99in+0\tabcolsep}}{\textcolor[HTML]{000000}{\fontsize{11}{11}\selectfont{5}}} & \multicolumn{1}{>{\centering}p{\dimexpr 0.68in+0\tabcolsep}}{\textcolor[HTML]{000000}{\fontsize{11}{11}\selectfont{-2.8}}} & \multicolumn{1}{>{\centering}p{\dimexpr 1.08in+0\tabcolsep}}{\textcolor[HTML]{000000}{\fontsize{11}{11}\selectfont{-5.3,\ -0.27}}} & \multicolumn{1}{>{\centering}p{\dimexpr 0.93in+0\tabcolsep}}{\textcolor[HTML]{000000}{\fontsize{11}{11}\selectfont{0.030}}} & \multicolumn{1}{>{\centering}p{\dimexpr 0.68in+0\tabcolsep}}{\textcolor[HTML]{000000}{\fontsize{11}{11}\selectfont{-2.7}}} & \multicolumn{1}{>{\centering}p{\dimexpr 1.08in+0\tabcolsep}}{\textcolor[HTML]{000000}{\fontsize{11}{11}\selectfont{-5.2,\ -0.14}}} & \multicolumn{1}{>{\centering}p{\dimexpr 0.93in+0\tabcolsep}}{\textcolor[HTML]{000000}{\fontsize{11}{11}\selectfont{0.038}}} & \multicolumn{1}{>{\centering}p{\dimexpr 0.68in+0\tabcolsep}}{\textcolor[HTML]{000000}{\fontsize{11}{11}\selectfont{-2.8}}} & \multicolumn{1}{>{\centering}p{\dimexpr 1.12in+0\tabcolsep}}{\textcolor[HTML]{000000}{\fontsize{11}{11}\selectfont{-5.4,\ -0.30}}} & \multicolumn{1}{>{\centering}p{\dimexpr 0.93in+0\tabcolsep}}{\textcolor[HTML]{000000}{\fontsize{11}{11}\selectfont{0.029}}} & \multicolumn{1}{>{\centering}p{\dimexpr 0.68in+0\tabcolsep}}{\textcolor[HTML]{000000}{\fontsize{11}{11}\selectfont{-2.7}}} & \multicolumn{1}{>{\centering}p{\dimexpr 1.12in+0\tabcolsep}}{\textcolor[HTML]{000000}{\fontsize{11}{11}\selectfont{-5.3,\ -0.19}}} & \multicolumn{1}{>{\centering}p{\dimexpr 0.93in+0\tabcolsep}}{\textcolor[HTML]{000000}{\fontsize{11}{11}\selectfont{0.035}}} \\





\multicolumn{1}{>{\raggedright}p{\dimexpr 1.99in+0\tabcolsep}}{\textcolor[HTML]{000000}{\fontsize{11}{11}\selectfont{6}}} & \multicolumn{1}{>{\centering}p{\dimexpr 0.68in+0\tabcolsep}}{\textcolor[HTML]{000000}{\fontsize{11}{11}\selectfont{-2.8}}} & \multicolumn{1}{>{\centering}p{\dimexpr 1.08in+0\tabcolsep}}{\textcolor[HTML]{000000}{\fontsize{11}{11}\selectfont{-6.0,\ 0.33}}} & \multicolumn{1}{>{\centering}p{\dimexpr 0.93in+0\tabcolsep}}{\textcolor[HTML]{000000}{\fontsize{11}{11}\selectfont{0.079}}} & \multicolumn{1}{>{\centering}p{\dimexpr 0.68in+0\tabcolsep}}{\textcolor[HTML]{000000}{\fontsize{11}{11}\selectfont{-2.6}}} & \multicolumn{1}{>{\centering}p{\dimexpr 1.08in+0\tabcolsep}}{\textcolor[HTML]{000000}{\fontsize{11}{11}\selectfont{-5.8,\ 0.52}}} & \multicolumn{1}{>{\centering}p{\dimexpr 0.93in+0\tabcolsep}}{\textcolor[HTML]{000000}{\fontsize{11}{11}\selectfont{0.10}}} & \multicolumn{1}{>{\centering}p{\dimexpr 0.68in+0\tabcolsep}}{\textcolor[HTML]{000000}{\fontsize{11}{11}\selectfont{-2.9}}} & \multicolumn{1}{>{\centering}p{\dimexpr 1.12in+0\tabcolsep}}{\textcolor[HTML]{000000}{\fontsize{11}{11}\selectfont{-6.1,\ 0.29}}} & \multicolumn{1}{>{\centering}p{\dimexpr 0.93in+0\tabcolsep}}{\textcolor[HTML]{000000}{\fontsize{11}{11}\selectfont{0.075}}} & \multicolumn{1}{>{\centering}p{\dimexpr 0.68in+0\tabcolsep}}{\textcolor[HTML]{000000}{\fontsize{11}{11}\selectfont{-2.7}}} & \multicolumn{1}{>{\centering}p{\dimexpr 1.12in+0\tabcolsep}}{\textcolor[HTML]{000000}{\fontsize{11}{11}\selectfont{-5.9,\ 0.46}}} & \multicolumn{1}{>{\centering}p{\dimexpr 0.93in+0\tabcolsep}}{\textcolor[HTML]{000000}{\fontsize{11}{11}\selectfont{0.094}}} \\





\multicolumn{1}{>{\raggedright}p{\dimexpr 1.99in+0\tabcolsep}}{\textcolor[HTML]{000000}{\fontsize{11}{11}\selectfont{7}}} & \multicolumn{1}{>{\centering}p{\dimexpr 0.68in+0\tabcolsep}}{\textcolor[HTML]{000000}{\fontsize{11}{11}\selectfont{-1.5}}} & \multicolumn{1}{>{\centering}p{\dimexpr 1.08in+0\tabcolsep}}{\textcolor[HTML]{000000}{\fontsize{11}{11}\selectfont{-4.2,\ 1.2}}} & \multicolumn{1}{>{\centering}p{\dimexpr 0.93in+0\tabcolsep}}{\textcolor[HTML]{000000}{\fontsize{11}{11}\selectfont{0.3}}} & \multicolumn{1}{>{\centering}p{\dimexpr 0.68in+0\tabcolsep}}{\textcolor[HTML]{000000}{\fontsize{11}{11}\selectfont{-1.4}}} & \multicolumn{1}{>{\centering}p{\dimexpr 1.08in+0\tabcolsep}}{\textcolor[HTML]{000000}{\fontsize{11}{11}\selectfont{-4.1,\ 1.3}}} & \multicolumn{1}{>{\centering}p{\dimexpr 0.93in+0\tabcolsep}}{\textcolor[HTML]{000000}{\fontsize{11}{11}\selectfont{0.3}}} & \multicolumn{1}{>{\centering}p{\dimexpr 0.68in+0\tabcolsep}}{\textcolor[HTML]{000000}{\fontsize{11}{11}\selectfont{-1.6}}} & \multicolumn{1}{>{\centering}p{\dimexpr 1.12in+0\tabcolsep}}{\textcolor[HTML]{000000}{\fontsize{11}{11}\selectfont{-4.3,\ 1.2}}} & \multicolumn{1}{>{\centering}p{\dimexpr 0.93in+0\tabcolsep}}{\textcolor[HTML]{000000}{\fontsize{11}{11}\selectfont{0.3}}} & \multicolumn{1}{>{\centering}p{\dimexpr 0.68in+0\tabcolsep}}{\textcolor[HTML]{000000}{\fontsize{11}{11}\selectfont{-1.6}}} & \multicolumn{1}{>{\centering}p{\dimexpr 1.12in+0\tabcolsep}}{\textcolor[HTML]{000000}{\fontsize{11}{11}\selectfont{-4.3,\ 1.2}}} & \multicolumn{1}{>{\centering}p{\dimexpr 0.93in+0\tabcolsep}}{\textcolor[HTML]{000000}{\fontsize{11}{11}\selectfont{0.3}}} \\





\multicolumn{1}{>{\raggedright}p{\dimexpr 1.99in+0\tabcolsep}}{\textcolor[HTML]{000000}{\fontsize{11}{11}\selectfont{8}}} & \multicolumn{1}{>{\centering}p{\dimexpr 0.68in+0\tabcolsep}}{\textcolor[HTML]{000000}{\fontsize{11}{11}\selectfont{-3.2}}} & \multicolumn{1}{>{\centering}p{\dimexpr 1.08in+0\tabcolsep}}{\textcolor[HTML]{000000}{\fontsize{11}{11}\selectfont{-5.6,\ -0.79}}} & \multicolumn{1}{>{\centering}p{\dimexpr 0.93in+0\tabcolsep}}{\textcolor[HTML]{000000}{\fontsize{11}{11}\selectfont{0.009}}} & \multicolumn{1}{>{\centering}p{\dimexpr 0.68in+0\tabcolsep}}{\textcolor[HTML]{000000}{\fontsize{11}{11}\selectfont{-3.1}}} & \multicolumn{1}{>{\centering}p{\dimexpr 1.08in+0\tabcolsep}}{\textcolor[HTML]{000000}{\fontsize{11}{11}\selectfont{-5.5,\ -0.67}}} & \multicolumn{1}{>{\centering}p{\dimexpr 0.93in+0\tabcolsep}}{\textcolor[HTML]{000000}{\fontsize{11}{11}\selectfont{0.013}}} & \multicolumn{1}{>{\centering}p{\dimexpr 0.68in+0\tabcolsep}}{\textcolor[HTML]{000000}{\fontsize{11}{11}\selectfont{-3.3}}} & \multicolumn{1}{>{\centering}p{\dimexpr 1.12in+0\tabcolsep}}{\textcolor[HTML]{000000}{\fontsize{11}{11}\selectfont{-5.8,\ -0.86}}} & \multicolumn{1}{>{\centering}p{\dimexpr 0.93in+0\tabcolsep}}{\textcolor[HTML]{000000}{\fontsize{11}{11}\selectfont{0.008}}} & \multicolumn{1}{>{\centering}p{\dimexpr 0.68in+0\tabcolsep}}{\textcolor[HTML]{000000}{\fontsize{11}{11}\selectfont{-3.3}}} & \multicolumn{1}{>{\centering}p{\dimexpr 1.12in+0\tabcolsep}}{\textcolor[HTML]{000000}{\fontsize{11}{11}\selectfont{-5.8,\ -0.82}}} & \multicolumn{1}{>{\centering}p{\dimexpr 0.93in+0\tabcolsep}}{\textcolor[HTML]{000000}{\fontsize{11}{11}\selectfont{0.009}}} \\





\multicolumn{1}{>{\raggedright}p{\dimexpr 1.99in+0\tabcolsep}}{\textcolor[HTML]{000000}{\fontsize{11}{11}\selectfont{EIMD\_2010\_quintile}}} & \multicolumn{1}{>{\centering}p{\dimexpr 0.68in+0\tabcolsep}}{\textcolor[HTML]{000000}{\fontsize{11}{11}\selectfont{}}} & \multicolumn{1}{>{\centering}p{\dimexpr 1.08in+0\tabcolsep}}{\textcolor[HTML]{000000}{\fontsize{11}{11}\selectfont{}}} & \multicolumn{1}{>{\centering}p{\dimexpr 0.93in+0\tabcolsep}}{\textcolor[HTML]{000000}{\fontsize{11}{11}\selectfont{}}} & \multicolumn{1}{>{\centering}p{\dimexpr 0.68in+0\tabcolsep}}{\textcolor[HTML]{000000}{\fontsize{11}{11}\selectfont{}}} & \multicolumn{1}{>{\centering}p{\dimexpr 1.08in+0\tabcolsep}}{\textcolor[HTML]{000000}{\fontsize{11}{11}\selectfont{}}} & \multicolumn{1}{>{\centering}p{\dimexpr 0.93in+0\tabcolsep}}{\textcolor[HTML]{000000}{\fontsize{11}{11}\selectfont{}}} & \multicolumn{1}{>{\centering}p{\dimexpr 0.68in+0\tabcolsep}}{\textcolor[HTML]{000000}{\fontsize{11}{11}\selectfont{}}} & \multicolumn{1}{>{\centering}p{\dimexpr 1.12in+0\tabcolsep}}{\textcolor[HTML]{000000}{\fontsize{11}{11}\selectfont{}}} & \multicolumn{1}{>{\centering}p{\dimexpr 0.93in+0\tabcolsep}}{\textcolor[HTML]{000000}{\fontsize{11}{11}\selectfont{}}} & \multicolumn{1}{>{\centering}p{\dimexpr 0.68in+0\tabcolsep}}{\textcolor[HTML]{000000}{\fontsize{11}{11}\selectfont{}}} & \multicolumn{1}{>{\centering}p{\dimexpr 1.12in+0\tabcolsep}}{\textcolor[HTML]{000000}{\fontsize{11}{11}\selectfont{}}} & \multicolumn{1}{>{\centering}p{\dimexpr 0.93in+0\tabcolsep}}{\textcolor[HTML]{000000}{\fontsize{11}{11}\selectfont{}}} \\





\multicolumn{1}{>{\raggedright}p{\dimexpr 1.99in+0\tabcolsep}}{\textcolor[HTML]{000000}{\fontsize{11}{11}\selectfont{1}}} & \multicolumn{1}{>{\centering}p{\dimexpr 0.68in+0\tabcolsep}}{\textcolor[HTML]{000000}{\fontsize{11}{11}\selectfont{—}}} & \multicolumn{1}{>{\centering}p{\dimexpr 1.08in+0\tabcolsep}}{\textcolor[HTML]{000000}{\fontsize{11}{11}\selectfont{—}}} & \multicolumn{1}{>{\centering}p{\dimexpr 0.93in+0\tabcolsep}}{\textcolor[HTML]{000000}{\fontsize{11}{11}\selectfont{}}} & \multicolumn{1}{>{\centering}p{\dimexpr 0.68in+0\tabcolsep}}{\textcolor[HTML]{000000}{\fontsize{11}{11}\selectfont{—}}} & \multicolumn{1}{>{\centering}p{\dimexpr 1.08in+0\tabcolsep}}{\textcolor[HTML]{000000}{\fontsize{11}{11}\selectfont{—}}} & \multicolumn{1}{>{\centering}p{\dimexpr 0.93in+0\tabcolsep}}{\textcolor[HTML]{000000}{\fontsize{11}{11}\selectfont{}}} & \multicolumn{1}{>{\centering}p{\dimexpr 0.68in+0\tabcolsep}}{\textcolor[HTML]{000000}{\fontsize{11}{11}\selectfont{—}}} & \multicolumn{1}{>{\centering}p{\dimexpr 1.12in+0\tabcolsep}}{\textcolor[HTML]{000000}{\fontsize{11}{11}\selectfont{—}}} & \multicolumn{1}{>{\centering}p{\dimexpr 0.93in+0\tabcolsep}}{\textcolor[HTML]{000000}{\fontsize{11}{11}\selectfont{}}} & \multicolumn{1}{>{\centering}p{\dimexpr 0.68in+0\tabcolsep}}{\textcolor[HTML]{000000}{\fontsize{11}{11}\selectfont{—}}} & \multicolumn{1}{>{\centering}p{\dimexpr 1.12in+0\tabcolsep}}{\textcolor[HTML]{000000}{\fontsize{11}{11}\selectfont{—}}} & \multicolumn{1}{>{\centering}p{\dimexpr 0.93in+0\tabcolsep}}{\textcolor[HTML]{000000}{\fontsize{11}{11}\selectfont{}}} \\





\multicolumn{1}{>{\raggedright}p{\dimexpr 1.99in+0\tabcolsep}}{\textcolor[HTML]{000000}{\fontsize{11}{11}\selectfont{2}}} & \multicolumn{1}{>{\centering}p{\dimexpr 0.68in+0\tabcolsep}}{\textcolor[HTML]{000000}{\fontsize{11}{11}\selectfont{0.66}}} & \multicolumn{1}{>{\centering}p{\dimexpr 1.08in+0\tabcolsep}}{\textcolor[HTML]{000000}{\fontsize{11}{11}\selectfont{-1.2,\ 2.5}}} & \multicolumn{1}{>{\centering}p{\dimexpr 0.93in+0\tabcolsep}}{\textcolor[HTML]{000000}{\fontsize{11}{11}\selectfont{0.5}}} & \multicolumn{1}{>{\centering}p{\dimexpr 0.68in+0\tabcolsep}}{\textcolor[HTML]{000000}{\fontsize{11}{11}\selectfont{0.63}}} & \multicolumn{1}{>{\centering}p{\dimexpr 1.08in+0\tabcolsep}}{\textcolor[HTML]{000000}{\fontsize{11}{11}\selectfont{-1.2,\ 2.5}}} & \multicolumn{1}{>{\centering}p{\dimexpr 0.93in+0\tabcolsep}}{\textcolor[HTML]{000000}{\fontsize{11}{11}\selectfont{0.5}}} & \multicolumn{1}{>{\centering}p{\dimexpr 0.68in+0\tabcolsep}}{\textcolor[HTML]{000000}{\fontsize{11}{11}\selectfont{0.65}}} & \multicolumn{1}{>{\centering}p{\dimexpr 1.12in+0\tabcolsep}}{\textcolor[HTML]{000000}{\fontsize{11}{11}\selectfont{-1.2,\ 2.5}}} & \multicolumn{1}{>{\centering}p{\dimexpr 0.93in+0\tabcolsep}}{\textcolor[HTML]{000000}{\fontsize{11}{11}\selectfont{0.5}}} & \multicolumn{1}{>{\centering}p{\dimexpr 0.68in+0\tabcolsep}}{\textcolor[HTML]{000000}{\fontsize{11}{11}\selectfont{0.61}}} & \multicolumn{1}{>{\centering}p{\dimexpr 1.12in+0\tabcolsep}}{\textcolor[HTML]{000000}{\fontsize{11}{11}\selectfont{-1.2,\ 2.4}}} & \multicolumn{1}{>{\centering}p{\dimexpr 0.93in+0\tabcolsep}}{\textcolor[HTML]{000000}{\fontsize{11}{11}\selectfont{0.5}}} \\





\multicolumn{1}{>{\raggedright}p{\dimexpr 1.99in+0\tabcolsep}}{\textcolor[HTML]{000000}{\fontsize{11}{11}\selectfont{3}}} & \multicolumn{1}{>{\centering}p{\dimexpr 0.68in+0\tabcolsep}}{\textcolor[HTML]{000000}{\fontsize{11}{11}\selectfont{0.77}}} & \multicolumn{1}{>{\centering}p{\dimexpr 1.08in+0\tabcolsep}}{\textcolor[HTML]{000000}{\fontsize{11}{11}\selectfont{-1.2,\ 2.7}}} & \multicolumn{1}{>{\centering}p{\dimexpr 0.93in+0\tabcolsep}}{\textcolor[HTML]{000000}{\fontsize{11}{11}\selectfont{0.4}}} & \multicolumn{1}{>{\centering}p{\dimexpr 0.68in+0\tabcolsep}}{\textcolor[HTML]{000000}{\fontsize{11}{11}\selectfont{0.65}}} & \multicolumn{1}{>{\centering}p{\dimexpr 1.08in+0\tabcolsep}}{\textcolor[HTML]{000000}{\fontsize{11}{11}\selectfont{-1.3,\ 2.6}}} & \multicolumn{1}{>{\centering}p{\dimexpr 0.93in+0\tabcolsep}}{\textcolor[HTML]{000000}{\fontsize{11}{11}\selectfont{0.5}}} & \multicolumn{1}{>{\centering}p{\dimexpr 0.68in+0\tabcolsep}}{\textcolor[HTML]{000000}{\fontsize{11}{11}\selectfont{0.76}}} & \multicolumn{1}{>{\centering}p{\dimexpr 1.12in+0\tabcolsep}}{\textcolor[HTML]{000000}{\fontsize{11}{11}\selectfont{-1.2,\ 2.7}}} & \multicolumn{1}{>{\centering}p{\dimexpr 0.93in+0\tabcolsep}}{\textcolor[HTML]{000000}{\fontsize{11}{11}\selectfont{0.4}}} & \multicolumn{1}{>{\centering}p{\dimexpr 0.68in+0\tabcolsep}}{\textcolor[HTML]{000000}{\fontsize{11}{11}\selectfont{0.64}}} & \multicolumn{1}{>{\centering}p{\dimexpr 1.12in+0\tabcolsep}}{\textcolor[HTML]{000000}{\fontsize{11}{11}\selectfont{-1.3,\ 2.6}}} & \multicolumn{1}{>{\centering}p{\dimexpr 0.93in+0\tabcolsep}}{\textcolor[HTML]{000000}{\fontsize{11}{11}\selectfont{0.5}}} \\





\multicolumn{1}{>{\raggedright}p{\dimexpr 1.99in+0\tabcolsep}}{\textcolor[HTML]{000000}{\fontsize{11}{11}\selectfont{4}}} & \multicolumn{1}{>{\centering}p{\dimexpr 0.68in+0\tabcolsep}}{\textcolor[HTML]{000000}{\fontsize{11}{11}\selectfont{0.58}}} & \multicolumn{1}{>{\centering}p{\dimexpr 1.08in+0\tabcolsep}}{\textcolor[HTML]{000000}{\fontsize{11}{11}\selectfont{-1.6,\ 2.7}}} & \multicolumn{1}{>{\centering}p{\dimexpr 0.93in+0\tabcolsep}}{\textcolor[HTML]{000000}{\fontsize{11}{11}\selectfont{0.6}}} & \multicolumn{1}{>{\centering}p{\dimexpr 0.68in+0\tabcolsep}}{\textcolor[HTML]{000000}{\fontsize{11}{11}\selectfont{0.53}}} & \multicolumn{1}{>{\centering}p{\dimexpr 1.08in+0\tabcolsep}}{\textcolor[HTML]{000000}{\fontsize{11}{11}\selectfont{-1.6,\ 2.7}}} & \multicolumn{1}{>{\centering}p{\dimexpr 0.93in+0\tabcolsep}}{\textcolor[HTML]{000000}{\fontsize{11}{11}\selectfont{0.6}}} & \multicolumn{1}{>{\centering}p{\dimexpr 0.68in+0\tabcolsep}}{\textcolor[HTML]{000000}{\fontsize{11}{11}\selectfont{0.58}}} & \multicolumn{1}{>{\centering}p{\dimexpr 1.12in+0\tabcolsep}}{\textcolor[HTML]{000000}{\fontsize{11}{11}\selectfont{-1.6,\ 2.7}}} & \multicolumn{1}{>{\centering}p{\dimexpr 0.93in+0\tabcolsep}}{\textcolor[HTML]{000000}{\fontsize{11}{11}\selectfont{0.6}}} & \multicolumn{1}{>{\centering}p{\dimexpr 0.68in+0\tabcolsep}}{\textcolor[HTML]{000000}{\fontsize{11}{11}\selectfont{0.53}}} & \multicolumn{1}{>{\centering}p{\dimexpr 1.12in+0\tabcolsep}}{\textcolor[HTML]{000000}{\fontsize{11}{11}\selectfont{-1.6,\ 2.7}}} & \multicolumn{1}{>{\centering}p{\dimexpr 0.93in+0\tabcolsep}}{\textcolor[HTML]{000000}{\fontsize{11}{11}\selectfont{0.6}}} \\





\multicolumn{1}{>{\raggedright}p{\dimexpr 1.99in+0\tabcolsep}}{\textcolor[HTML]{000000}{\fontsize{11}{11}\selectfont{5}}} & \multicolumn{1}{>{\centering}p{\dimexpr 0.68in+0\tabcolsep}}{\textcolor[HTML]{000000}{\fontsize{11}{11}\selectfont{0.52}}} & \multicolumn{1}{>{\centering}p{\dimexpr 1.08in+0\tabcolsep}}{\textcolor[HTML]{000000}{\fontsize{11}{11}\selectfont{-1.5,\ 2.5}}} & \multicolumn{1}{>{\centering}p{\dimexpr 0.93in+0\tabcolsep}}{\textcolor[HTML]{000000}{\fontsize{11}{11}\selectfont{0.6}}} & \multicolumn{1}{>{\centering}p{\dimexpr 0.68in+0\tabcolsep}}{\textcolor[HTML]{000000}{\fontsize{11}{11}\selectfont{0.49}}} & \multicolumn{1}{>{\centering}p{\dimexpr 1.08in+0\tabcolsep}}{\textcolor[HTML]{000000}{\fontsize{11}{11}\selectfont{-1.5,\ 2.5}}} & \multicolumn{1}{>{\centering}p{\dimexpr 0.93in+0\tabcolsep}}{\textcolor[HTML]{000000}{\fontsize{11}{11}\selectfont{0.6}}} & \multicolumn{1}{>{\centering}p{\dimexpr 0.68in+0\tabcolsep}}{\textcolor[HTML]{000000}{\fontsize{11}{11}\selectfont{0.53}}} & \multicolumn{1}{>{\centering}p{\dimexpr 1.12in+0\tabcolsep}}{\textcolor[HTML]{000000}{\fontsize{11}{11}\selectfont{-1.4,\ 2.5}}} & \multicolumn{1}{>{\centering}p{\dimexpr 0.93in+0\tabcolsep}}{\textcolor[HTML]{000000}{\fontsize{11}{11}\selectfont{0.6}}} & \multicolumn{1}{>{\centering}p{\dimexpr 0.68in+0\tabcolsep}}{\textcolor[HTML]{000000}{\fontsize{11}{11}\selectfont{0.51}}} & \multicolumn{1}{>{\centering}p{\dimexpr 1.12in+0\tabcolsep}}{\textcolor[HTML]{000000}{\fontsize{11}{11}\selectfont{-1.4,\ 2.5}}} & \multicolumn{1}{>{\centering}p{\dimexpr 0.93in+0\tabcolsep}}{\textcolor[HTML]{000000}{\fontsize{11}{11}\selectfont{0.6}}} \\





\multicolumn{1}{>{\raggedright}p{\dimexpr 1.99in+0\tabcolsep}}{\textcolor[HTML]{000000}{\fontsize{11}{11}\selectfont{Sodium\ (mg)\ diet\ only}}} & \multicolumn{1}{>{\centering}p{\dimexpr 0.68in+0\tabcolsep}}{\textcolor[HTML]{000000}{\fontsize{11}{11}\selectfont{}}} & \multicolumn{1}{>{\centering}p{\dimexpr 1.08in+0\tabcolsep}}{\textcolor[HTML]{000000}{\fontsize{11}{11}\selectfont{}}} & \multicolumn{1}{>{\centering}p{\dimexpr 0.93in+0\tabcolsep}}{\textcolor[HTML]{000000}{\fontsize{11}{11}\selectfont{}}} & \multicolumn{1}{>{\centering}p{\dimexpr 0.68in+0\tabcolsep}}{\textcolor[HTML]{000000}{\fontsize{11}{11}\selectfont{0.00}}} & \multicolumn{1}{>{\centering}p{\dimexpr 1.08in+0\tabcolsep}}{\textcolor[HTML]{000000}{\fontsize{11}{11}\selectfont{0.00,\ 0.00}}} & \multicolumn{1}{>{\centering}p{\dimexpr 0.93in+0\tabcolsep}}{\textcolor[HTML]{000000}{\fontsize{11}{11}\selectfont{0.043}}} & \multicolumn{1}{>{\centering}p{\dimexpr 0.68in+0\tabcolsep}}{\textcolor[HTML]{000000}{\fontsize{11}{11}\selectfont{}}} & \multicolumn{1}{>{\centering}p{\dimexpr 1.12in+0\tabcolsep}}{\textcolor[HTML]{000000}{\fontsize{11}{11}\selectfont{}}} & \multicolumn{1}{>{\centering}p{\dimexpr 0.93in+0\tabcolsep}}{\textcolor[HTML]{000000}{\fontsize{11}{11}\selectfont{}}} & \multicolumn{1}{>{\centering}p{\dimexpr 0.68in+0\tabcolsep}}{\textcolor[HTML]{000000}{\fontsize{11}{11}\selectfont{0.00}}} & \multicolumn{1}{>{\centering}p{\dimexpr 1.12in+0\tabcolsep}}{\textcolor[HTML]{000000}{\fontsize{11}{11}\selectfont{0.00,\ 0.00}}} & \multicolumn{1}{>{\centering}p{\dimexpr 0.93in+0\tabcolsep}}{\textcolor[HTML]{000000}{\fontsize{11}{11}\selectfont{0.029}}} \\





\multicolumn{1}{>{\raggedright}p{\dimexpr 1.99in+0\tabcolsep}}{\textcolor[HTML]{000000}{\fontsize{11}{11}\selectfont{Epcnt\_4}}} & \multicolumn{1}{>{\centering}p{\dimexpr 0.68in+0\tabcolsep}}{\textcolor[HTML]{000000}{\fontsize{11}{11}\selectfont{}}} & \multicolumn{1}{>{\centering}p{\dimexpr 1.08in+0\tabcolsep}}{\textcolor[HTML]{000000}{\fontsize{11}{11}\selectfont{}}} & \multicolumn{1}{>{\centering}p{\dimexpr 0.93in+0\tabcolsep}}{\textcolor[HTML]{000000}{\fontsize{11}{11}\selectfont{}}} & \multicolumn{1}{>{\centering}p{\dimexpr 0.68in+0\tabcolsep}}{\textcolor[HTML]{000000}{\fontsize{11}{11}\selectfont{}}} & \multicolumn{1}{>{\centering}p{\dimexpr 1.08in+0\tabcolsep}}{\textcolor[HTML]{000000}{\fontsize{11}{11}\selectfont{}}} & \multicolumn{1}{>{\centering}p{\dimexpr 0.93in+0\tabcolsep}}{\textcolor[HTML]{000000}{\fontsize{11}{11}\selectfont{}}} & \multicolumn{1}{>{\centering}p{\dimexpr 0.68in+0\tabcolsep}}{\textcolor[HTML]{000000}{\fontsize{11}{11}\selectfont{-0.01}}} & \multicolumn{1}{>{\centering}p{\dimexpr 1.12in+0\tabcolsep}}{\textcolor[HTML]{000000}{\fontsize{11}{11}\selectfont{-0.05,\ 0.03}}} & \multicolumn{1}{>{\centering}p{\dimexpr 0.93in+0\tabcolsep}}{\textcolor[HTML]{000000}{\fontsize{11}{11}\selectfont{0.6}}} & \multicolumn{1}{>{\centering}p{\dimexpr 0.68in+0\tabcolsep}}{\textcolor[HTML]{000000}{\fontsize{11}{11}\selectfont{-0.02}}} & \multicolumn{1}{>{\centering}p{\dimexpr 1.12in+0\tabcolsep}}{\textcolor[HTML]{000000}{\fontsize{11}{11}\selectfont{-0.06,\ 0.02}}} & \multicolumn{1}{>{\centering}p{\dimexpr 0.93in+0\tabcolsep}}{\textcolor[HTML]{000000}{\fontsize{11}{11}\selectfont{0.3}}} \\

\ascline{1pt}{000000}{1-13}



\end{longtable}



\arrayrulecolor[HTML]{000000}

\global\setlength{\arrayrulewidth}{\Oldarrayrulewidth}

\global\setlength{\tabcolsep}{\Oldtabcolsep}

\renewcommand*{\arraystretch}{1}

\global\setlength{\Oldarrayrulewidth}{\arrayrulewidth}

\global\setlength{\Oldtabcolsep}{\tabcolsep}

\setlength{\tabcolsep}{0pt}

\renewcommand*{\arraystretch}{1.5}



\providecommand{\ascline}[3]{\noalign{\global\arrayrulewidth #1}\arrayrulecolor[HTML]{#2}\cline{#3}}

\begin{longtable}[c]{|p{0.75in}|p{0.75in}}



\ascline{2pt}{666666}{1-2}

\multicolumn{1}{>{\raggedright}p{\dimexpr 0.75in+0\tabcolsep}}{\textcolor[HTML]{000000}{\fontsize{11}{11}\selectfont{Model}}} & \multicolumn{1}{>{\raggedleft}p{\dimexpr 0.75in+0\tabcolsep}}{\textcolor[HTML]{000000}{\fontsize{11}{11}\selectfont{AIC}}} \\

\ascline{2pt}{666666}{1-2}\endhead



\multicolumn{1}{>{\raggedright}p{\dimexpr 0.75in+0\tabcolsep}}{\textcolor[HTML]{000000}{\fontsize{11}{11}\selectfont{No\ sodium\ or\ UPF}}} & \multicolumn{1}{>{\raggedleft}p{\dimexpr 0.75in+0\tabcolsep}}{\textcolor[HTML]{000000}{\fontsize{11}{11}\selectfont{28,435.99}}} \\





\multicolumn{1}{>{\raggedright}p{\dimexpr 0.75in+0\tabcolsep}}{\textcolor[HTML]{000000}{\fontsize{11}{11}\selectfont{Sodium\ only}}} & \multicolumn{1}{>{\raggedleft}p{\dimexpr 0.75in+0\tabcolsep}}{\textcolor[HTML]{000000}{\fontsize{11}{11}\selectfont{28,430.71}}} \\





\multicolumn{1}{>{\raggedright}p{\dimexpr 0.75in+0\tabcolsep}}{\textcolor[HTML]{000000}{\fontsize{11}{11}\selectfont{Epcnt\ only}}} & \multicolumn{1}{>{\raggedleft}p{\dimexpr 0.75in+0\tabcolsep}}{\textcolor[HTML]{000000}{\fontsize{11}{11}\selectfont{28,437.53}}} \\





\multicolumn{1}{>{\raggedright}p{\dimexpr 0.75in+0\tabcolsep}}{\textcolor[HTML]{000000}{\fontsize{11}{11}\selectfont{Sodium\ and\ UPF}}} & \multicolumn{1}{>{\raggedleft}p{\dimexpr 0.75in+0\tabcolsep}}{\textcolor[HTML]{000000}{\fontsize{11}{11}\selectfont{28,431.29}}} \\

\ascline{2pt}{666666}{1-2}



\end{longtable}



\arrayrulecolor[HTML]{000000}

\global\setlength{\arrayrulewidth}{\Oldarrayrulewidth}

\global\setlength{\tabcolsep}{\Oldtabcolsep}

\renewcommand*{\arraystretch}{1}

\begin{verbatim}
## [1] 0.02400091
\end{verbatim}

we find that the lowest AIC is given by the model without UPF!! Though
all the models with UPF have a lower aic than the model without.

\hypertarget{summary-of-results}{%
\subsection{Summary of Results}\label{summary-of-results}}

There is a table with summary values for the key variables across the
dataset.

Statistical analysis of the key variables shows the change in all the
variables between the two time periods.

Confounding variables are analysed and show if there has been a
significant change in the balance of the populations.

Regression shows a degree of association between the BP and UPF intake
by weight and by energy. It also shows the same for sodium intake.

Using Anova analysis of different multi variable regression models the
key variables are significant for sodium in several models, and
sometimes for UPF.

\hypertarget{results-conclusion}{%
\subsection{Results Conclusion}\label{results-conclusion}}

The percentage by energy of NOVA group 4 foods decreased from 2008 to
2019. The mean sodium intake in mg decreased. The systolic BP has
decreased.

There is a correlation between systolic BP and sodium intake. There is a
correlation between systolic BP and UPF intake.

The regression models identify that age and sex are statistically
significant contributors to the BP and that bmi, educfinh, and IMD are
also.

The regression models identify that sodium intake is an important
contributor to any optimal model. That UPF intake is no longer
significant, but still has some effect. \newpage

\hypertarget{discussion}{%
\section{Discussion}\label{discussion}}

\hypertarget{introduction-to-discussion}{%
\subsection{Introduction to
Discussion}\label{introduction-to-discussion}}

This section will consider what the results mean. That will include how
the context of the literature influences understanding of the values.
The limitations of the study will be considered. opportunities for
further research, and to influence policy will be considered.

\hypertarget{discussion-of-results}{%
\subsection{Discussion of results}\label{discussion-of-results}}

The study provides a number of results which will are first considered
individually before being brought together to support the development of
the dissertation.

\hypertarget{data}{%
\subsubsection{Data}\label{data}}

The data is well collected and comprehensive. There have been several
changes over the course of the study. These changes have meant that
collating the data was more than just bringing all the numbers together.

In addition the differing take up rates between different groups in each
of the cohorts meant that the numbers from each cohort are not
comparable. This is overcome by using weighting factors to balance the
datasets. This needs adjusting every time there is a new group added to
the collected data set.

The analysis using these weighted datasets is performed using ``survey''
which is a software package used in r studio.

The table (x) of data shows how the key data items alter over the survey
years. In particular demonstrating the lack of BP data from year 11.

\hypertarget{comparative-data}{%
\subsubsection{comparative Data}\label{comparative-data}}

\hypertarget{univariable-regression}{%
\subsubsection{univariable regression}\label{univariable-regression}}

\hypertarget{of-bp}{%
\paragraph{of BP}\label{of-bp}}

\hypertarget{of-other-comparisons}{%
\paragraph{of other comparisons}\label{of-other-comparisons}}

\hypertarget{multivariable-models}{%
\subsubsection{multivariable models}\label{multivariable-models}}

\hypertarget{analysis-for-bp}{%
\paragraph{analysis for BP}\label{analysis-for-bp}}

\hypertarget{analysis-for-sodium}{%
\paragraph{analysis for Sodium}\label{analysis-for-sodium}}

\hypertarget{analysis-for-upf_4}{%
\paragraph{analysis for UPF\_4}\label{analysis-for-upf_4}}

\hypertarget{limitations-of-study}{%
\subsection{Limitations of Study}\label{limitations-of-study}}

\hypertarget{ideas-for-further-research}{%
\subsection{Ideas for further
research}\label{ideas-for-further-research}}

I will divide these suggestions into quantitative and qualitative.
Within the quantitative there are biomedical

\hypertarget{quantitative}{%
\subsubsection{Quantitative}\label{quantitative}}

There is scope for more research based on this data set. Within this
same biomedical paradigm there are whole range of variables which can be
compared against the clinical and biochemical outcomes.

\hypertarget{mixed-and-qualitative}{%
\subsubsection{Mixed and Qualitative}\label{mixed-and-qualitative}}

The richness of the quantitative data in this survey calls for its use
within an approach allowing more detailed description and in depth
assessment with participants.

It could also be used as a template for studies smaller in geographical
scope, but more in depth as cross over studies collecting both
quantitative and qualitative data.

Modelling research has allowed projections to be made using

\hypertarget{ideas-for-policy}{%
\subsection{Ideas for policy}\label{ideas-for-policy}}

Policy is an `upstream' approach.

Ideas include legislation to reduce UPF use, this might be by pricing,
or other approaches.

Health promotion policy needs to match policy activity. People who know
that UPF is bad, are more likely to accept policy limiting availability.

\newpage

\hypertarget{conclusion}{%
\section{Conclusion}\label{conclusion}}

\newpage

\hypertarget{bibliography-1}{%
\section{Bibliography}\label{bibliography-1}}

\hypertarget{references}{%
\section{References}\label{references}}

\hypertarget{refs}{}
\begin{CSLReferences}{0}{0}
\end{CSLReferences}

\hypertarget{appendix-appendix}{%
\section*{(APPENDIX) Appendix}\label{appendix-appendix}}
\addcontentsline{toc}{section}{(APPENDIX) Appendix}

\hypertarget{more-information}{%
\section{More information}\label{more-information}}

This will be Appendix A.

\hypertarget{one-more-thing}{%
\section{One more thing}\label{one-more-thing}}

This will be Appendix B. \# Appendices

\end{document}
